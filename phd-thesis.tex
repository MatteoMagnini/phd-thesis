\documentclass[12pt,a4paper,openright,twoside]{book}
\usepackage[utf8]{inputenc}
\usepackage{phd-thesis}
\usepackage{code-lstlistings}
\usepackage{notes}
\usepackage{shortcuts}
\usepackage{myacronyms}

\mainlinespacing{1.241} % line spacing in mainmatter, comment to default (1)

\begin{document}
	
\frontmatter
%!TeX root = phd-thesis.tex
\title{Title}
\author{Candidate Name Here}
\date{\today}

\newgeometry{margin=0.8in}
\begin{titlepage}
	\begin{center}
		% \vspace*{0.2cm}

		\large
		\textbf{ALMA MATER STUDIORUM -- UNIVERSITÀ DI BOLOGNA \\ DISI Dipartimento di Informatica: Scienza e Ingegneria}
		\\
		\noindent\hrulefill
		\vspace{0.4cm}

		\Large
		Dottorato di Ricerca in \\
		Computer Science and Engineering

		\vspace{0.4cm}

		Ciclo XXXVIII

		\vspace{0.4cm}

		Settore Scientifico Disciplinare: ING-INF/05

		Settore Concorsuale: 09/H1

		\Huge
		\vspace{3cm}
		\textbf{
			Your Fancy Title Here
		}

		{\Large{
		\vspace{3cm}

		\textit{Candidato:\\}
		\centering
		Dott. Matteo Magnini}
		\\}
		\large
		\vspace{2.5cm}
		\begin{minipage}[t]{0.64\textwidth}
			\begin{flushleft}
				\textit{Coordinatrice Dottorato:}
				\\
				\textbf{Prof.ssa Ilaria Bartolini}
			\end{flushleft}
		\end{minipage}
		\begin{minipage}[t]{0.34\textwidth}
			\begin{flushright}
				\textit{Supervisore:}
				\\
				\textbf{Prof.} \textbf{Andrea Omicini}
				\\
				\vspace{0.4cm}
				\textit{Co-supervisore:}
				\\
				\textbf{Prof.} \textbf{Enrico Denti}
			\end{flushright}

		\end{minipage}\\

		\vfill
		\noindent\hrulefill
		\vspace{0.3cm}
		\Large

		Esame finale anno 2025
	\end{center}
\end{titlepage}
\restoregeometry

\begin{abstract}	
% Max 2000 characters, strict.
    \Ac{AI} is one of the most promising mankind's inventions in recent history, and it already has a significant impact on our daily lives.
    %
    During the last decade \ac{NeSy} \ac{AI}, that aims to develop systems that integrate symbolic and sub-symbolic \ac{AI}, has gained increasing attention.
    %
    The exploitation of both symbolic and sub-symbolic aspects of \ac{AI} is crucial to develop systems that can overcome the limitations of the two paradigms.
    %
    Beyond all the possible research branches, \ac{SKI} and \ac{SKE} are two major areas that can have a key role in the development of intelligent systems.
    %
    At the writing of this thesis, we are experiencing a de facto Copernican revolution that is not limited to the \ac{AI} field but also involves the whole society: the advent of \acp{LLM}.

    This thesis focuses on \ac{NeSy} \ac{AI}, in particular on \ac{SKI}, \ac{SKE} and on the engineering of intelligent software systems.
    %
    First, we reconcile the most relevant definitions and concepts after a comprehensive overview of the background and the state of the art in the field of \ac{NeSy} \ac{AI},
    %
    We then identify the challenges associated with \ac{NeSy} \ac{AI} and how we can address them.
    %
    In particular, present original contributions in terms of methodologies, algorithms, and tools for \ac{SKI} and \ac{SKE} that enable us to further develop \ac{NeSy} \ac{AI} systems.
    %
    Finally, we discuss how to design intelligent systems that exploit the synergy between symbolic and sub-symbolic \ac{AI}, especially in critical domains such as fairness and healthcare.
    %
    Ultimately, we present intelligent systems that can autonomously learn.



\end{abstract}

\begin{dedication} % this is optional
``Whatever it takes`` -- Mario Draghi, July 26th 2012
\end{dedication}

\begin{acknowledgements} % this is optional
    %
    First and foremost, I would like to thank my parents, Barbara and Marcello, for their unconditional love and support.
    %
    Both of them thought me the importance of hard work and perseverance.
    %
    My dad made me passionate in science and technology, my mom the interest for history.
    %
    I thank my grandparents, Mirka, Franca and Marco, for their love and presence.
    %
    Without them all, I would have not started a Ph.D., at least, not here in Italy.
    %
    The disgusting lack of financial support from the Italian government for the scientific research has always been -- since the infamous Gentile's reform -- a curse for the development of the country.

    I would like to thank my supervisor, Prof. Andrea Omicini, for his knowledge, guidance, and support.
    %
    Beyond anything else, I really appreciated the freedom he gave me during my journey.
    %
    I thank my senior colleague, and de facto co-supervisor, prof. Giovanni Ciatto for his daily support and guidance especially during my first year as a research fellow and in the first year of my Ph.D.
    %

\end{acknowledgements}

%----------------------------------------------------------------------------------------
\tableofcontents   
\listoffigures     % (optional) comment if empty
\lstlistoflistings % (optional) comment if empty
%----------------------------------------------------------------------------------------

\mainmatter

%----------------------------------------------------------------------------------------
\chapter{Introduction}
\label{ch:introduction}
%----------------------------------------------------------------------------------------

Write your intro here.
\sidenote{Add sidenotes in this way. They are named after the author of the thesis}


\section{Research Background and Context}\label{sec:research-background-and-context}


\section{Overview and Contributions}\label{sec:overview-and-contributions}


\section{Structure of the Thesis}\label{sec:structure-of-the-thesis}


\note{Racall to describe the structure of the paper}


%----------------------------------------------------------------------------------------
%----------------------------------------------------------------------------------------

\part{Background}\label{part:background}

%----------------------------------------------------------------------------------------
%----------------------------------------------------------------------------------------

\chapter{Intelligent Systems}\label{ch:intelligent-systems}

\section{What is intelligence?}\label{sec:what-is-intelligence}

\subsection{Intelligence in humans}\label{subsec:intelligence-in-humans}

\subsection{Intelligence in animals}\label{subsec:intelligence-in-animals}

\section{Intelligence in computer science}\label{sec:intelligence-in-computer-science}

\subsection{Reasoning}\label{subsec:reasoning}

\subsection{Planning}\label{subsec:planning}

\subsection{Adapting}\label{subsec:adapting}

\subsection{Swarm intelligence}\label{subsec:swarm-intelligence}

\subsection{Learning}\label{subsec:learning}

\subsubsection{\Acl{RL}}\label{subsubsec:rl}

%----------------------------------------------------------------------------------------

\chapter{\Acl{AI}}\label{ch:ai}

\section{Symbolic \ac{AI}}\label{sec:symbolic-ai}

\subsection{Logic}\label{subsec:logic}

\subsection{Expert Systems}\label{subsec:expert-systems}

\subsection{Trees and other algorithms}\label{subsec:trees-and-other-algorithms}

\section{Sub-symbolic \ac{AI}}\label{sec:sub-symbolic-ai}

\subsection{\Aclp{NN}}\label{subsec:neural-networks}

\section{\Acl{NeSy} \ac{AI}}\label{sec:nesy-ai}

\subsection{\Acl{SKI}}\label{subsec:ski}

\subsubsection{Structuring}\label{subsubsec:structuring}

\subsubsection{Learning}\label{subsubsec:learning}

\subsubsection{Embedding}\label{subsec:embedding}

\subsection{\Acl{SKE}}\label{subsec:ske}

\subsubsection{Local explanations}\label{subsubsec:local-explanations}

\subsubsection{Global explanations}\label{subsubsec:global-explanations}

%----------------------------------------------------------------------------------------

\chapter{\Aclp{LLM}}\label{ch:llm}

\section{Architectures}\label{sec:llm-architectures}

\subsection{Transformer}\label{subsec:transformer}

\subsection{Attention}\label{subsec:attention}

\section{Fine-tuning}\label{sec:llm-fine-tuning}

\section{\Acp{RAG}}\label{sec:rag}

\section{Limitations and challenges}\label{sec:limitations-and-challenges}

\subsection{Resources}\label{subsec:resources}

\subsection{Data and privacy}\label{subsec:data-and-privacy}

%----------------------------------------------------------------------------------------
%----------------------------------------------------------------------------------------

\part{Engineering of \ac{SKI} \& \ac{SKE}}\label{part:engineering-of-ski-ske}

%----------------------------------------------------------------------------------------
%----------------------------------------------------------------------------------------

\chapter{Common Patterns}\label{ch:common-patterns}

\section{Knowledge Representation}\label{sec:knowledge-representation}

\subsection{Logic Rules}\label{subsec:logic-rules}

\subsection{\Aclp{KG}}\label{subsec:kg}

\subsection{Free text}\label{subsec:free-text}

\section{From symbolic to sub-symbolic}\label{sec:from-symbolic-to-sub-symbolic}

\subsection{Fuzzyfication}\label{subsec:fuzzyfication}

\subsection{Delegation}\label{subsec:delegation}

\section{From sub-symbolic to symbolic}\label{sec:from-sub-symbolic-to-symbolic}

\subsection{Surrogate models}\label{subsec:surrogate-models}

%----------------------------------------------------------------------------------------

\chapter{\Acl{PSyKI}}\label{ch:psyki}

\section{Implementation}\label{sec:implementation}

\subsection{Goals}\label{subsec:goals}

\subsection{Architecture}\label{subsec:architecture}

\subsection{Details}\label{subsec:details}

\section{Available \ac{SKI} methods}\label{sec:available-ski-methods}

\subsection{\Acl{KBANN}}\label{subsec:kbann}

\subsection{\Acl{KINS}}\label{subsec:kins}

\subsection{\Acl{KILL}}\label{subsec:kill}

\section{\Acl{QoS} for \ac{SKI}}\label{sec:qos}

\section{Fairness}\label{sec:fairness}

\subsection{\Acl{FaUCI}}\label{subsec:fauci}


%----------------------------------------------------------------------------------------
%----------------------------------------------------------------------------------------

\part{Engineering of Intelligent Systems}\label{part:engineering-of-intelligent-systems}

%----------------------------------------------------------------------------------------
%----------------------------------------------------------------------------------------

\chapter{Autonomous Learning}\label{ch:autonomous-learning}

\section{Goals and Challenges}\label{sec:goals-and-challenges}

\section{Applications}\label{sec:applications}

\subsection{Exact Learning with \acp{LLM}}\label{subsec:exact-learning-with-ac{llm}}

\subsection{\Acl{LLM4KB}}\label{subsec:llm4kb}

%----------------------------------------------------------------------------------------

\chapter{Conclusions}\label{ch:conclusions}

\section{Discussion}\label{sec:discussion}

\section{Future Work}\label{sec:future-work}

%----------------------------------------------------------------------------------------
% BIBLIOGRAPHY
%----------------------------------------------------------------------------------------

\backmatter

\part*{}

\nocite{*} % comment this to only show the referenced entries from the .bib file
\bibliographystyle{alpha}
\bibliography{phd-thesis}

\end{document}