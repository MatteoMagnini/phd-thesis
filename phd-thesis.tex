\documentclass[12pt,a4paper,openright,twoside]{book}
\usepackage[utf8]{inputenc}
\usepackage{phd-thesis}
\usepackage{code-lstlistings}
\usepackage{notes}
\usepackage{shortcuts}
\usepackage{myacronyms}

\mainlinespacing{1.241} % line spacing in mainmatter, comment to default (1)

\begin{document}
	
\frontmatter
%!TeX root = phd-thesis.tex
\title{Symbolic Knowledge Injection \& Extraction for Autonomous Learning}
\author{Matteo Magnini}
\date{\today}

\newgeometry{margin=0.8in}
\begin{titlepage}
	\begin{center}
		% \vspace*{0.2cm}

		\large
		\textbf{ALMA MATER STUDIORUM -- UNIVERSITÀ DI BOLOGNA \\ DISI Dipartimento di Informatica: Scienza e Ingegneria}
		\\
		\noindent\hrulefill
		\vspace{0.4cm}

		\Large
		Doctor of Philosophy in \\
		Computer Science and Engineering

		\vspace{0.4cm}

		Cycle XXXVIII

		\vspace{0.4cm}
		%TODO: update as soon as Ilaria provides the offcial template
		Settore Scientifico Disciplinare: ING-INF/05

		Settore Concorsuale: 09/H1

		\Huge
		\vspace{3cm}
		\textbf{
			Symbolic Knowledge Injection \& Extraction for Autonomous Learning
		}

		{\Large{
		\vspace{3cm}

		\textit{Candidate:\\}
		\centering
		Matteo Magnini}
		\\}
		\large
		\vspace{2.5cm}
		\begin{minipage}[t]{0.64\textwidth}
			\begin{flushleft}
				\textit{PhD coordinator:}
				\\
				\textbf{Prof. Ilaria Bartolini}
			\end{flushleft}
		\end{minipage}
		\begin{minipage}[t]{0.34\textwidth}
			\begin{flushright}
				\textit{Supervisor:}
				\\
				\textbf{Prof.} \textbf{Andrea Omicini}
				\\
				\vspace{0.4cm}
				\textit{Co-supervisor:}
				\\
				\textbf{Prof.} \textbf{Enrico Denti}
			\end{flushright}

		\end{minipage}\\

		\vfill
		\noindent\hrulefill
		\vspace{0.3cm}
		\Large
		%TODO: update as soon as Ilaria provides the offcial template

		Esame finale anno 2025
	\end{center}
\end{titlepage}
\restoregeometry

\begin{abstract}
    %
    \Ac{AI} is one of the most promising mankind's inventions in recent history, and it has already achieved a significant impact on our daily lives.
    %
    During the last decade \ac{NeSy} \ac{AI}, that aims to develop systems that integrate \emph{symbolic} and \emph{sub-symbolic} \ac{AI}, has gained increasing attention.
    %
    The exploitation of both symbolic and sub-symbolic aspects of \ac{AI} is crucial to develop systems that can overcome the limitations of the two paradigms.
    %
    Beyond all the possible research branches, \ac{SKI} and \ac{SKE} are two major areas that can have a key role in the development of \emph{intelligent systems}.
    %
    At the writing of this thesis, we are experiencing a de facto Copernican revolution that is not limited to the \ac{AI} field but also involves the whole society: the advent of the \acp{LLM}.

    This thesis focuses on \ac{NeSy} \ac{AI}, in particular on \ac{SKI}, \ac{SKE} and on the engineering of \emph{intelligent software systems}.
    %
    First, we introduce the reader to what intelligence is, both in carbonium -- humans and animals -- and in silico---computers.
    %
    We reconcile the most relevant definitions and concepts after a comprehensive overview of the background and the state of the art in the field of \ac{NeSy} \ac{AI}.
    %
    Then, we identify the challenges associated with \ac{NeSy} \ac{AI} and we present the research agenda we followed to address them.
    %
    In particular, present original contributions in terms of methodologies, algorithms, and tools for \ac{SKI} and \ac{SKE} that enable us to further develop \ac{NeSy} \ac{AI} systems.
    %
    Finally, we discuss how to design intelligent systems that exploit the synergy between symbolic and sub-symbolic \ac{AI}, especially in critical domains such as fairness and healthcare.
    %
    Ultimately, we present intelligent systems that can autonomously learn.

    \sloppypar
    Keywords -- \emph{\acl{AI}, \acl{NeSy} \ac{AI}, \acl{SKI}, \acl{SKE}, \aclp{LLM}, \aclp{MAS}, Intelligent Systems, Autonomous Learning Systems}

\end{abstract}

\begin{dedication} % this is optional
    %
    ``The work of each individual contributes to a totality, and so becomes an undying part of the totality.
    %
    That totality of human lives—past and present and to come—forms a tapestry that has been in existence now for many tens of thousands of years and has been growing more elaborate and, on the whole, more beautiful in all that time.
    %
    % Even the Spacers are an offshoot of the tapestry and they, too, add to the elaborateness and beauty of the pattern.
    %
    [\dots] An individual life is one thread in the tapestry and what is one thread compared to the whole?
    %
    Daneel, keep your mind fixed firmly on the tapestry and do not let the trailing off of a single thread affect you.'' --- Isaac Asimov, \emph{Robots and Empire}
    %
\end{dedication}

\begin{acknowledgements} % this is optional

\end{acknowledgements}

%----------------------------------------------------------------------------------------
\tableofcontents   
\listoffigures     % (optional) comment if empty
\lstlistoflistings % (optional) comment if empty
%----------------------------------------------------------------------------------------

\mainmatter

%----------------------------------------------------------------------------------------
\chapter{Introduction}
\label{ch:introduction}
%----------------------------------------------------------------------------------------

\section{Research background and context}\label{sec:research-background-and-context}

\section{Overview and contributions}\label{sec:overview-and-contributions}

\section{Structure of the thesis}\label{sec:structure-of-the-thesis}

%----------------------------------------------------------------------------------------
%----------------------------------------------------------------------------------------

\part{Background}\label{part:background}

%----------------------------------------------------------------------------------------
%----------------------------------------------------------------------------------------

\chapter{Intelligent Systems}\label{ch:intelligent-systems}

\section{What is intelligence?}\label{sec:what-is-intelligence}

\subsection{Intelligence in animals}\label{subsec:intelligence-in-animals}

\subsection{Intelligence in humans}\label{subsec:intelligence-in-humans}

\section{Intelligence in computer science}\label{sec:intelligence-in-computer-science}

\subsection{Reasoning}\label{subsec:reasoning}

\subsection{Autonomy}\label{subsec:autonomy}

\subsection{Planning}\label{subsec:planning}

\subsection{Adapting}\label{subsec:adapting}

\section{Learning}\label{sec:learning}

\subsection{Actively learning}\label{subsec:actively-learning}

\subsection{\Acl{PAC} learning}\label{subsec:pac-learning}

\subsection{Autonomous learning}\label{subsec:autonomous-learning}

\subsubsection{\Acl{RL}}\label{subsubsec:rl}

\subsection{\Aclp{MAS}}\label{subsec:mas}

\subsubsection{Swarm intelligence}\label{subsubsec:swarm-intelligence}

\subsubsection{\Acl{MARL}}\label{subsubsec:marl}

\subsection{General learning}\label{subsec:general-learning}

%----------------------------------------------------------------------------------------

\chapter{\Acl{AI}}\label{ch:ai}

\section{Symbolic \ac{AI}}\label{sec:symbolic-ai}

\subsection{\Acl{CL}}\label{subsec:cl}

\subsubsection{\Acl{DL}}\label{subsubsec:dl}

\subsubsection{Ontologies}\label{subsubsec:ontologies}

\subsubsection{Horn logic}\label{subsubsec:horn-logic}

\subsubsection{Higher-order logic}\label{subsubsec:higher-order-logic}

\subsection{Expert Systems}\label{subsec:expert-systems}

\subsection{Trees and other algorithms}\label{subsec:trees-and-other-algorithms}

\subsection{Limits of symbolic \ac{AI}}\label{subsec:limits-of-symbolic-ai}

\section{Sub-symbolic \ac{AI}}\label{sec:sub-symbolic-ai}

\subsection{Random forests}\label{subsec:random-forests}

\subsection{Bayesian methods}\label{subsec:bayesian-methods}

\subsection{Evolutionary algorithms}\label{subsec:evolutionary-algorithms}

\subsection{\Aclp{SVM}}\label{subsec:svm}

\subsection{\Aclp{NN}}\label{subsec:neural-networks}

\subsection{Limits of sub-symbolic \ac{AI}}\label{subsec:limits-of-sub-symbolic-ai}

%----------------------------------------------------------------------------------------

\chapter{\Acl{NeSy} \ac{AI}}\label{ch:nesy-ai}

\section{\Acl{SKI}}\label{sec:ski}

\subsection{Motivations and goals}\label{subsec:ski-motivations-and-goals}

\subsection{What to inject}\label{subsec:what-to-inject}

\subsection{How to inject}\label{subsec:how-to-inject}

\subsubsection{Structuring}\label{subsubsec:structuring}

\subsubsection{Learning}\label{subsubsec:learning}

\subsubsection{Embedding}\label{subsubsec:embedding}

\subsection{Limitations and challenges of \ac{SKI}}\label{subsec:limitations-and-challenges-of-ski}

\section{\Acl{SKE}}\label{sec:ske}

\subsection{Motivations and goals}\label{subsec:ske-motivations-and-goals}

\subsection{How to extract}\label{subsec:how-to-extract}

\subsubsection{Decompositional \ac{SKE}}\label{subsubsec:decompositional-ske}

\subsubsection{Pedagocial \ac{SKE}}\label{subsubsec:pedagogical-ske}

\subsubsection{Local explanations}\label{subsubsec:local-explanations}

\subsubsection{Global explanations}\label{subsubsec:global-explanations}

\subsection{Limitations and challenges of \ac{SKE}}\label{subsec:limitations-and-challenges-of-ske}

%----------------------------------------------------------------------------------------

\chapter{\Aclp{LLM}}\label{ch:llm}

\section{Architectures}\label{sec:llm-architectures}

\subsection{Transformer}\label{subsec:transformer}

\subsection{Attention}\label{subsec:attention}

\section{Fine-tuning}\label{sec:llm-fine-tuning}

\section{\Acp{RAG}}\label{sec:rag}

\section{Limitations and challenges}\label{sec:limitations-and-challenges}

\subsection{Resources}\label{subsec:resources}

\subsection{Data and privacy}\label{subsec:data-and-privacy}

\subsection{Hallucinations}\label{subsec:hallucinations}

\subsection{Stochastic parrot or something more?}\label{subsec:stochastic-parrot-or-something-more}

%----------------------------------------------------------------------------------------
%----------------------------------------------------------------------------------------

\part{Engineering of \ac{SKI} \& \ac{SKE}}\label{part:engineering-of-ski-ske}

%----------------------------------------------------------------------------------------
%----------------------------------------------------------------------------------------

\chapter{Common patterns}\label{ch:common-patterns}

\section{Knowledge representation}\label{sec:knowledge-representation}

\subsection{Logic rules}\label{subsec:logic-rules}

\subsection{\Aclp{KG}}\label{subsec:kg}

\subsection{Free text}\label{subsec:free-text}

\section{From symbolic to sub-symbolic}\label{sec:from-symbolic-to-sub-symbolic}

\subsection{Fuzzyfication}\label{subsec:fuzzyfication}

\subsection{Delegation}\label{subsec:delegation}

\section{From sub-symbolic to symbolic}\label{sec:from-sub-symbolic-to-symbolic}

\subsection{Surrogate models}\label{subsec:surrogate-models}

\subsection{De-fuzzyfication}\label{subsec:unfuzzyfication}

%----------------------------------------------------------------------------------------

\chapter{\Acl{PSyKI}}\label{ch:psyki}

\section{Implementation}\label{sec:implementation}

\subsection{Goals}\label{subsec:goals}

\subsection{Architecture}\label{subsec:architecture}

\subsection{Details}\label{subsec:details}

\section{Available \ac{SKI} methods}\label{sec:available-ski-methods}

\subsection{\Acl{KBANN}}\label{subsec:kbann}

\subsection{\Acl{KINS}}\label{subsec:kins}

\subsection{\Acl{KILL}}\label{subsec:kill}

\section{\Acl{QoS} for \ac{SKI}}\label{sec:qos}

\section{Fairness}\label{sec:fairness}

\subsection{\Acl{FaUCI}}\label{subsec:fauci}

%----------------------------------------------------------------------------------------
%----------------------------------------------------------------------------------------

\part{Engineering of intelligent systems}\label{part:engineering-of-intelligent-systems}

%----------------------------------------------------------------------------------------
%----------------------------------------------------------------------------------------

\chapter{\Ac{NeSy} \ac{AI} for real world applications}\label{ch:nesy-ai-for-real-world-applications}

\section{Motivations}\label{sec:nesy-ai-motivations}

\section{Goals and challenges}\label{sec:nesy-ai-goals-and-challenges}

\section{Applications}\label{sec:nesy-ai-applications}

\subsection{\Ac{SKE} for explainable nutritional recommenders}\label{subsec:ske-for-explainable-nutritional-recommenders}

\subsection{A general-purpose protocol for multi-agent based explanations}\label{subsec:a-general-purpose-protocol-for-multi-agent-based-explanations}

\subsection{\Acl{NeSy} \ac{AI} for supporting chronic disease diagnosis and monitoring}\label{subsec:nesy-ai-for-supporting-chronic-disease-diagnosis-and-monitoring}

\subsection{\Ac{LLM}-based solutions for healthcare chatbots: a comparative analysis}\label{subsec:llm-based-solutions-for-healthcare-chatbots-a-comparative-analysis}

\subsection{Open-source small language models for personal medical assistant chatbots}\label{subsec:open-source-small-language-models-for-personal-medical-assistant-chatbots}

\subsection{Applying \acl{RAG} on open \acp{LLM} for a medical chatbot supporting hypertensive patients}\label{subsec:applying-rag-on-open-llm-for-a-medical-chatbot-supporting-hypertensive-patients}

%----------------------------------------------------------------------------------------

\chapter{Autonomous learning systems}\label{ch:autonomous-learning-systems}

\section{Motivations}\label{sec:motivations}

\section{Goals and challenges}\label{sec:goals-and-challenges}

\section{Applications}\label{sec:applications}

\subsection{Actively learning ontologies from \acp{LLM}}\label{subsec:exact-learning-with-ac{llm}}

\subsection{\Aclp{LLM} as oracles for instantiating ontologies with domain-specific knowledge}\label{subsec:llm-as-oracles-for-instantiating-ontologies-with-domain-specific-knowledge}

%----------------------------------------------------------------------------------------

\chapter{Conclusions}\label{ch:conclusions}

\section{Discussion}\label{sec:discussion}

\section{Future work}\label{sec:future-work}

%----------------------------------------------------------------------------------------
% BIBLIOGRAPHY
%----------------------------------------------------------------------------------------

\backmatter

\part*{}

\nocite{*} % comment this to only show the referenced entries from the .bib file
\bibliographystyle{alpha}
\bibliography{phd-thesis}

\end{document}