\documentclass[12pt,a4paper,openright,twoside]{book}
\usepackage[utf8]{inputenc}
\usepackage{phd-thesis}
\usepackage{code-lstlistings}
\usepackage{notes}
\usepackage{shortcuts}
\usepackage{myacronyms}

\mainlinespacing{1.241} % line spacing in mainmatter, comment to default (1)

\begin{document}
	
\frontmatter
%!TeX root = phd-thesis.tex
\title{Symbolic Knowledge Injection \& Extraction for Autonomous Learning}
\author{Matteo Magnini}
\date{\today}

\newgeometry{margin=0.8in}
\begin{titlepage}
	\begin{center}
		% \vspace*{0.2cm}

		\large
		\textbf{ALMA MATER STUDIORUM -- UNIVERSITÀ DI BOLOGNA \\ DISI Dipartimento di Informatica: Scienza e Ingegneria}
		\\
		\noindent\hrulefill
		\vspace{0.4cm}

		\Large
		Doctor of Philosophy in \\
		Computer Science and Engineering

		\vspace{0.4cm}

		Cycle XXXVIII

		\vspace{0.4cm}
		%TODO: update as soon as Ilaria provides the offcial template
		Settore Scientifico Disciplinare: ING-INF/05

		Settore Concorsuale: 09/H1

		\Huge
		\vspace{3cm}
		\textbf{
			Symbolic Knowledge Injection \& Extraction for Autonomous Learning
		}

		{\Large{
		\vspace{3cm}

		\textit{Candidate:\\}
		\centering
		Matteo Magnini}
		\\}
		\large
		\vspace{2.5cm}
		\begin{minipage}[t]{0.64\textwidth}
			\begin{flushleft}
				\textit{PhD coordinator:}
				\\
				\textbf{Prof. Ilaria Bartolini}
			\end{flushleft}
		\end{minipage}
		\begin{minipage}[t]{0.34\textwidth}
			\begin{flushright}
				\textit{Supervisor:}
				\\
				\textbf{Prof.} \textbf{Andrea Omicini}
				\\
				\vspace{0.4cm}
				\textit{Co-supervisor:}
				\\
				\textbf{Prof.} \textbf{Enrico Denti}
			\end{flushright}

		\end{minipage}\\

		\vfill
		\noindent\hrulefill
		\vspace{0.3cm}
		\Large
		%TODO: update as soon as Ilaria provides the offcial template

		Esame finale anno 2025
	\end{center}
\end{titlepage}
\restoregeometry

\begin{abstract}	
% Max 2000 characters, strict.
    \Ac{AI} is one of the most promising mankind's inventions in recent history, and it already has a significant impact on our daily lives.
    %
    During the last decade \ac{NeSy} \ac{AI}, that aims to develop systems that integrate symbolic and sub-symbolic \ac{AI}, has gained increasing attention.
    %
    The exploitation of both symbolic and sub-symbolic aspects of \ac{AI} is crucial to develop systems that can overcome the limitations of the two paradigms.
    %
    Beyond all the possible research branches, \ac{SKI} and \ac{SKE} are two major areas that can have a key role in the development of intelligent systems.
    %
    At the writing of this thesis, we are experiencing a de facto Copernican revolution that is not limited to the \ac{AI} field but also involves the whole society: the advent of \acp{LLM}.

    This thesis focuses on \ac{NeSy} \ac{AI}, in particular on \ac{SKI}, \ac{SKE} and on the engineering of intelligent software systems.
    %
    First, we reconcile the most relevant definitions and concepts after a comprehensive overview of the background and the state of the art in the field of \ac{NeSy} \ac{AI},
    %
    We then identify the challenges associated with \ac{NeSy} \ac{AI} and how we can address them.
    %
    In particular, present original contributions in terms of methodologies, algorithms, and tools for \ac{SKI} and \ac{SKE} that enable us to further develop \ac{NeSy} \ac{AI} systems.
    %
    Finally, we discuss how to design intelligent systems that exploit the synergy between symbolic and sub-symbolic \ac{AI}, especially in critical domains such as fairness and healthcare.
    %
    Ultimately, we present intelligent systems that can autonomously learn.



\end{abstract}

\begin{dedication} % this is optional
``Whatever it takes`` -- Mario Draghi, July 26th 2012
\end{dedication}

\begin{acknowledgements} % this is optional
    %
    First and foremost, I would like to thank my parents, Barbara and Marcello, for their unconditional love and support.
    %
    Both of them thought me the importance of hard work and perseverance.
    %
    My dad made me passionate in science and technology, my mom the interest for history.
    %
    I thank my grandparents, Mirka, Franca and Marco, for their love and presence.
    %
    Without them all, I would have not started a Ph.D., at least, not here in Italy.
    %
    The disgusting lack of financial support from the Italian government for the scientific research has always been -- since the infamous Gentile's reform -- a curse for the development of the country.

    I would like to thank my supervisor, Prof. Andrea Omicini, for his knowledge, guidance, and support.
    %
    Beyond anything else, I really appreciated the freedom he gave me during my journey.
    %
    I thank my senior colleague, and de facto co-supervisor, prof. Giovanni Ciatto for his daily support and guidance especially during my first year as a research fellow and in the first year of my Ph.D.
    %


\end{acknowledgements}

%----------------------------------------------------------------------------------------
\tableofcontents   
\listoffigures     % (optional) comment if empty
\lstlistoflistings % (optional) comment if empty
%----------------------------------------------------------------------------------------

\mainmatter

%----------------------------------------------------------------------------------------
\chapter{Introduction}
\label{chap:introduction}
%----------------------------------------------------------------------------------------

Write your intro here.
\sidenote{Add sidenotes in this way. They are named after the author of the thesis}

\paragraph{Structure of the Thesis}

\note{Racall to describe the structure of the paper}

\part{First Part}

\chapter{State of the art}

I suggest referencing stuff as follows: \cref{fig:random-image} or \Cref{fig:random-image}

\begin{figure}
    \centering
    \includegraphics[width=.8\linewidth]{figures/random-image.pdf}
    \caption{Some random image}
    \label{fig:random-image}
\end{figure}

\section{Some cool topic}

\part{Second Part}

\chapter{Contribution}

You may also put some code snippet (which is NOT float by default), eg: \cref{lst:random-code}.

\lstinputlisting[float,language=Java,label={lst:random-code}]{listings/HelloWorld.java}

\section{Fancy formulas here}

%----------------------------------------------------------------------------------------
% BIBLIOGRAPHY
%----------------------------------------------------------------------------------------

\backmatter

\part*{}

\nocite{*} % comment this to only show the referenced entries from the .bib file
\bibliographystyle{alpha}
\bibliography{phd-thesis}

\end{document}