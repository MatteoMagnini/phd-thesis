%! Author = matteomagnini
%! Date = 05/03/25

%----------------------------------------------------------------------------------------
\chapter[Artificial Intelligence]{\Glsentrylong{AI}}
\label{ch:ai}
\minitoc
%----------------------------------------------------------------------------------------

\section{Overview}\label{sec:ai-overview}
%
The wide range of \gls{AI} techniques used to be divided into two main categories: \emph{symbolic} and \emph{sub-symbolic} \gls{AI}.
%
The thing that distinguishes the two categories is the way they \emph{represent knowledge} and how they process it.
%
No intelligence can exist without knowledge and no computation can occur in lack of representation.
%
In the rest of the thesis, we will use the term symbolic (resp., sub-symbolic) \gls{AI} and symbolic (resp., sub-symbolic) \gls{KR} almost interchangeably.
%
Symbolic \gls{AI} is based on \emph{symbols}, which come with a \emph{meaning} and could be manipulated according to the formalism and rules of a given \gls{AI} system.
%
On the other hand, sub-symbolic \gls{AI} is based on a numerical representation -- a.k.a., sub-symbolic -- where the numbers are not directly interpretable.
%
Numbers are technically symbols, but numbers, arrays and their functions are not recognised as means for symbolic \gls{KR}.
%
According to Van Gelder~\cite{DBLP:conf/ogai/Gelder90}, in order to be considered symbolic, \gls{KR} approaches must:
%
\begin{requirements}
    %
    \item \label{itm:symbolic-req-1} involve a set of symbols;
    %
    \item \label{itm:symbolic-req-2} the symbols can be combined following a set of grammatical rules;
    %
    \item \label{itm:symbolic-req-3} elementary symbols and combinations of symbols can be assigned a meaning.
    %
\end{requirements}


\paragraph{Local vs. distributed}
%
Multidimensional arrays are the fundamental building block of sub-symbolic data representation.
%
Formally, a $D$-order array is an ordered container of real numbers, where $D$ indicates the number of indices required to access each element.
%
We refer to 1-order arrays as \emph{vectors}, 2-order arrays as \emph{matrices}, and arrays of order greater than two as \emph{tensors}.
%
In sub-symbolic tasks based on arrays, information is typically conveyed both by the values stored in the array and their position within it.
%
The dimensions of the array -- denoted as $(d_1 \times \dots \times d_D)$ -- also play a crucial role, as sub-symbolic systems are usually designed to operate on arrays of fixed shape.
%
That is, the values of $d_1, \dots, d_D$ are chosen at design time and remain unchanged thereafter.
%
This violates \Cref{itm:symbolic-req-2} above; accordingly, we define sub-symbolic \gls{KR} as the task of encoding information into rigid numeric arrays.
%
\emph{Local} and \emph{distributed} representations are two key modes for encoding data into such arrays.
%
In local representations, each entry in the array corresponds to a well-defined concept from the target domain---its semantic meaning is clear and independent.
%
In distributed representations, by contrast, individual values carry little or no standalone meaning: their interpretation depends on the configuration of values across a neighbourhood in the indexing space.
%
Consequently, while the exact location of values is largely irrelevant in local representations, it becomes essential in distributed ones.
%
Notably, distributed representations violate \Cref{itm:symbolic-req-3}, and for this reason, recent literature often labels as \emph{sub-symbolic} those predictors that rely on distributed encoding of data.


\section{Symbolic \Gls{AI}}\label{sec:symbolic-ai}
%
Symbolic \gls{AI} has been regarded as crucial since \gls{AI}'s inception.
%
Symbolic \gls{KR} offers enhanced flexibility, expressiveness, and intelligibility, being interpretable both by machines and by humans.

\paragraph{Intentional vs. extensional}
%
In formal logic, one may define concepts either \emph{extensionally} or \emph{intensionally}.
%
Extensional definitions are direct representations of data.
%
For example, the set of square numbers admits the extensional definition $\{0,1,4,9,16,\dots\}$ by listing every member explicitly.
%
Conversely, an \emph{intensional} definition is an indirect representation of data.
%
In \gls{FOL}, this corresponds to defining a relation via a formula; for instance, the set of square numbers can be defined as $\{\,x\mid \exists n\in\mathbb{Z}\,(x = n^2)\}$ which succinctly encodes an infinite extension with a single schema.
%
Recursive intensional predicates further enhance expressivity: for example, the ancestor relation can be axiomatized by $\mathit{Ancestor}(x,y)\;\Leftrightarrow\;\mathit{Parent}(x,y)\;\lor\;\exists z\,[\,\mathit{Parent}(x,z)\wedge\mathit{Ancestor}(z,y)\,]$ allowing a compact representation of an infinite set of pairs with a finite rule.
%
In formal logic, intensional definitions are prized for their ability to model potentially unbounded domains within finite logical formalisms.


\paragraph{Expressiveness vs. tractability}
%
Tractability addresses the theoretical question of whether a logic reasoner can determine the truth of a given formula within feasible time and space bounds.
%
The answer is deeply tied to the specific reasoning algorithm and the logic's formal properties.
%
Depending on the features a logic provides -- such as quantifiers, function symbols, or recursive definitions -- it may be more or less expressive.
%
The higher the expressiveness, the more complex the problems that can be represented and reasoned about, but this also increases the computational burden.
%
This well-known phenomenon is often referred to as the expressiveness/tractability trade-off~\cite{DBLP:journals/jlp/CadoliS93,BRACHMAN2004327,DBLP:journals/ci/LevesqueB87}.
%
In practice, highly expressive logics make it easier for human users to model rich domains, often requiring fewer and more concise formulas.
%
However, this comes at the cost of automated inference, which may become computationally intractable, undecidable, or non-terminating in the general case.
%
To mitigate this issue, various fragments and extensions of \gls{FOL} have been identified, each providing different tradeoffs between what can be expressed and what can be decided efficiently.


\subsection{\Glsentrylong{FOL}}\label{subsec:first-order-logic}
%
\Gls{FOL} is a general-purpose formalism that underpins most symbolic \gls{KR} systems.
%
It enables both human and computational agents to model entities and their interrelations through predicates and terms within a defined domain of discourse.
%
Its syntax comprises variables (quantified explicitly or implicitly), constants, function symbols, and predicate symbols, which are combined via logical operators such as conjunction (\(\wedge\)), disjunction (\(\vee\)), implication (\(\rightarrow\)), and equivalence (\(\leftrightarrow\)).
%
\Gls{FLO} allows for both \emph{extensional} and \emph{intensional} definitions.
%
Recursive intensional definitions, in particular, are powerful, enabling finite representations of infinite sets.
%
Despite its flexibility, \gls{FOL} is semi-decidable in general: there is no algorithm that can determine the truth of every \gls{FOL} formula in finite time, which limits its use in systems requiring guaranteed termination~\cite{DBLP:conf/dlog/2003handbook}.


\subsection{Horn logic}\label{subsec:horn-logic}
%
Horn logic is a significant subset of \gls{FOL}, offering a balanced trade-off between theoretical expressiveness and practical tractability~\cite{DBLP:journals/jcss/Makowsky87}.
%
It is built around the concept of \emph{Horn clauses}~\cite{DBLP:journals/jsyml/Horn51}, which are formulas in \gls{FOL} that exclude quantifiers and consist of a disjunction of predicates, with at most one non-negated literal.
%
Alternatively, a Horn clause can be expressed as an implication where the consequent is a single predicate and the antecedent is a conjunction of predicates: \(h \gets b_1, \dots, b_n\).
%
Here, \(\gets\) denotes logical implication (from right to left), commas represent logical conjunctions, and \(b_i\) as well as \(h\) are predicates of arbitrary arity, potentially containing \gls{FOL} terms such as variables, constants, or functions.

Horn clauses can be interpreted as \emph{if-then} rules written in reverse order, where only conjunctions of predicates are allowed in the antecedent.
%
In essence, Horn logic is a constrained subset of \gls{FOL} characterized by the following limitations:
%
\begin{inlinelist}
%
    \item formulas are reduced to clauses, containing only predicates, conjunctions, and a single implication operator;
    %
    \item operators such as \(\lor\), \(\leftrightarrow\), or \(\neg\) (negation) are not allowed;
    %
    \item variables are implicitly quantified; and
    %
    \item terms behave as they do in \gls{FOL}.
    %
\end{inlinelist}


\subsection{Datalog}\label{subsec:datalog}
%
Datalog is a declarative query language and a restricted subset of \gls{FOL}, designed for deductive databases and knowledge representation~\cite{DBLP:journals/jcss/AjtaiG94}.
%
It represents knowledge using function-free \gls{Horn} clauses, as defined in \Cref{subsec:horn-logic}.
%
This restriction eliminates the use of function symbols, thereby forbidding structured terms such as recursive data structures.
%
As a result, Datalog is well-suited for applications requiring finite and decidable reasoning, as the absence of function symbols ensures termination of inference algorithms.
%
Similar to \gls{Horn} logic, Datalog’s knowledge bases consist of sets of function-free \gls{Horn} clauses, which are interpreted as rules and facts.
%
Rules in Datalog follow the form \(h \gets b_1, \dots, b_n\), where \(h\) is the head of the rule and \(b_1, \dots, b_n\) are the body predicates.
%
Unlike general \gls{FOL}, Datalog does not allow disjunctions, negations, or explicit quantifiers, as variables are implicitly universally quantified.
%
Datalog is widely used in areas such as \glspl{KG}, semantic web technologies, and database systems, where efficient reasoning over large datasets is required.
%
Its simplicity and computational efficiency make it a practical choice for symbolic \gls{AI} tasks that demand tractable reasoning.


\subsection{\Glsentrylong{DL}}\label{subsec:dl}

\subsection{Ontologies and \glsentrylong{KG}}\label{subsec:ontologies-and-kg}

\subsection{Trees and other algorithms}\label{subsec:trees-and-other-algorithms}

\subsection{Limits of symbolic \Gls{AI}}\label{subsec:limits-of-symbolic-ai}



\section{Sub-symbolic \Gls{AI}}\label{sec:sub-symbolic-ai}

\subsection{Random forests}\label{subsec:random-forests}

\subsection{Bayesian methods}\label{subsec:bayesian-methods}

\subsection{Evolutionary algorithms}\label{subsec:evolutionary-algorithms}

\subsection{\Glsentrylongpl{SVM}}\label{subsec:svm}

\subsection{\Glsentrylongpl{NN}}\label{subsec:neural-networks}

\subsection{Limits of sub-symbolic \Gls{AI}}\label{subsec:limits-of-sub-symbolic-ai}