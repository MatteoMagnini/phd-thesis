%! Author = matteomagnini
%! Date = 05/03/25

%----------------------------------------------------------------------------------------
\chapter[Artificial Intelligence]{\Glsentrylong{AI}}
\label{ch:ai}
\mtcaddchapter
\minitoc
%----------------------------------------------------------------------------------------

\section{Overview}\label{sec:ai-overview}
%
The wide range of \gls{AI} techniques used to be divided into two main categories: \emph{symbolic} and \emph{sub-symbolic} \Gls{AI}.
%
The thing that distinguishes the two categories is the way they \emph{represent knowledge} and how they process it.
%
No intelligence can exist without knowledge and no computation can occur in lack of representation.
%


\section{Symbolic \Gls{AI}}\label{sec:symbolic-ai}
%


\subsection{\Glsentrylong{CL}}\label{subsec:cl}

\subsection{\Glsentrylong{DL}}\label{subsec:dl}

\subsection{Ontologies}\label{subsec:ontologies}

\subsection{Horn logic}\label{subsec:horn-logic}

\subsection{Higher-order logic}\label{subsec:higher-order-logic}

\subsection{Expert Systems}\label{subsec:expert-systems}

\subsection{Trees and other algorithms}\label{subsec:trees-and-other-algorithms}

\subsection{Limits of symbolic \Gls{AI}}\label{subsec:limits-of-symbolic-ai}

\section{Sub-symbolic \Gls{AI}}\label{sec:sub-symbolic-ai}

\subsection{Random forests}\label{subsec:random-forests}

\subsection{Bayesian methods}\label{subsec:bayesian-methods}

\subsection{Evolutionary algorithms}\label{subsec:evolutionary-algorithms}

\subsection{\Glsentrylongpl{SVM}}\label{subsec:svm}

\subsection{\Glsentrylongpl{NN}}\label{subsec:neural-networks}

\subsection{Limits of sub-symbolic \Gls{AI}}\label{subsec:limits-of-sub-symbolic-ai}