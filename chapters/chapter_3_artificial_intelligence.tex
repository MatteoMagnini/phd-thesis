%! Author = matteomagnini
%! Date = 05/03/25

%----------------------------------------------------------------------------------------
\chapter[Artificial Intelligence]{\Glsentrylong{AI}}
\label{ch:ai}
\minitoc
%----------------------------------------------------------------------------------------

\section{Overview}\label{sec:ai-overview}
%
The wide range of \gls{AI} techniques used to be divided into two main categories: \emph{symbolic} and \emph{sub-symbolic} \gls{AI}.
%
The thing that distinguishes the two categories is the way they \emph{represent knowledge} and how they process it.
%
No intelligence can exist without knowledge and no computation can occur in lack of representation.
%
In the rest of the thesis, we will use the term symbolic (resp., sub-symbolic) \gls{AI} and symbolic (resp., sub-symbolic) \gls{KR} almost interchangeably.
%
Symbolic \gls{AI} is based on \emph{symbols}, which come with a \emph{meaning} and could be manipulated according to the formalism and rules of a given \gls{AI} system.
%
On the other hand, sub-symbolic \gls{AI} is based on a numerical representation -- a.k.a., sub-symbolic -- where the numbers are not directly interpretable.
%
\note{Talk more about the two systems in general}
%
In the following sections, we will explore symbolic and sub-symbolic \gls{AI} in more detail.


\section{Symbolic \Gls{AI}}\label{sec:symbolic-ai}
%
Symbolic \gls{AI} has been regarded as crucial since \gls{AI}'s inception.
%
Symbolic \gls{KR} offers enhanced flexibility, expressiveness, and intelligibility, being interpretable both by machines and by humans.

\paragraph{Intentional vs. extensional}
%
In formal logic, one may define concepts either \emph{extensionally} or \emph{intensionally}.
%
Extensional definitions are direct representations of data.
%
For example, the set of square numbers admits the extensional definition $\{0,1,4,9,16,\dots\}$ by listing every member explicitly.
%
Conversely, an \emph{intensional} definition is an indirect representation of data.
%
In \gle{FOL}, this corresponds to defining a relation via a formula; for instance, the set of square numbers can be defined as $\{\,x\mid \exists n\in\mathbb{Z}\,(x = n^2)\}$ which succinctly encodes an infinite extension with a single schema.
%
Recursive intensional predicates further enhance expressivity: for example, the ancestor relation can be axiomatized by $\mathit{Ancestor}(x,y)\;\Leftrightarrow\;\mathit{Parent}(x,y)\;\lor\;\exists z\,[\,\mathit{Parent}(x,z)\wedge\mathit{Ancestor}(z,y)\,]$ allowing a compact representation of an infinite set of pairs with a finite rule.
%
In formal logic, intensional definitions are prized for their ability to model potentially unbounded domains within finite logical formalisms.


Historically, most \gls{KR} formalisms and their enabling technologies have been rooted in \gls{CL}.
%
Notable examples, for instance, include \gls{FOL}, \gls{HL}, \gls{DL}, and ontologies.
%
Increased expressiveness often comes at the cost of higher computational complexity, giving rise to the well-known \emph{expressiveness vs. tractability} trade-off~\cite{DBLP:conf/dlog/2003handbook}.
%
To balance these concerns, various fragments and extensions of \gls{FOL} have been identified, each providing different tradeoffs between what can be expressed and what can be decided efficiently.

\paragraph{Expressiveness vs. tractability}
%
\note{complete this}


\subsection{\Glsentrylong{FOL}}\label{subsec:first-order-logic}

\subsection{\Glsentrylong{DL}}\label{subsec:dl}

\subsection{Ontologies}\label{subsec:ontologies}

\subsection{Horn logic}\label{subsec:horn-logic}

\subsection{Expert Systems}\label{subsec:expert-systems}

\subsection{Trees and other algorithms}\label{subsec:trees-and-other-algorithms}

\subsection{Limits of symbolic \Gls{AI}}\label{subsec:limits-of-symbolic-ai}

\section{Sub-symbolic \Gls{AI}}\label{sec:sub-symbolic-ai}

\subsection{Random forests}\label{subsec:random-forests}

\subsection{Bayesian methods}\label{subsec:bayesian-methods}

\subsection{Evolutionary algorithms}\label{subsec:evolutionary-algorithms}

\subsection{\Glsentrylongpl{SVM}}\label{subsec:svm}

\subsection{\Glsentrylongpl{NN}}\label{subsec:neural-networks}

\subsection{Limits of sub-symbolic \Gls{AI}}\label{subsec:limits-of-sub-symbolic-ai}