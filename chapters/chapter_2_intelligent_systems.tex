%! Author = matteomagnini
%! Date = 05/03/25

%----------------------------------------------------------------------------------------
\chapter{Intelligent Systems}
\label{ch:intelligent-systems}
\mtcaddchapter
\minitoc
%----------------------------------------------------------------------------------------

\section{What is intelligence?}\label{sec:what-is-intelligence}

Intelligence is a concept that encompasses a wide range of abilities and characteristics of single \emph{individuals} or \emph{groups}.
%
From an \emph{evolutionary} perspective, intelligence characterises some animal species -- and organisms belonging to other biological kingdoms -- from simple forms of life.
%
In the history of our planet, carbon-based life evolved from unicellular organisms to multicellular organisms, and ultimately to complex organisms with specialized cells and tissues.
%
Some of these organisms (e.g., insects) developed skills -- such as \emph{navigation}, \emph{communication}, \emph{self-organisation}, \emph{adaptation}, and so on -- that \emph{are perceived} as intelligent abilities.
%
More evolved individuals (e.g., mammals) developed even more complex forms of intelligence, such as \emph{planning}, \emph{reasoning}, \emph{learning}, etc.


Intelligence emerged as a result of the evolutionary process and from the interaction of organisms with each others and their environment.
%
However, it is us -- as humans -- to attribute the label of \emph{intelligent} to some behaviours and not to others.
%
Indeed, it is said that ``intelligence is in the eye of the beholder''.
%
In this sense, we can say that intelligence is not an absolute concept, but it should be considered under a relative perspective.


In much more recent history, the carbon-based life was not the only one to ``manifest'' intelligent behaviours.
%
With the invention of the computer, also machines can perform tasks that we consider intelligent.
%
To distinguish between the intelligence of living beings and that of machines, we refer to the former as \emph{natural intelligence} and to the latter as \emph{\gls{AI}}.


In this chapter, we briefly talk about intelligence in animals and humans (~\Cref{subsec:intelligence-in-animals,subsec:intelligence-in-humans}).
%
Then, we introduce \gls{AI} and its most relevant subfields (~\Cref{sec:intelligence-in-computer-science}).
%
Finally, we focus on the core topic of this thesis: \emph{learning} and \emph{autonomous learning} (~\Cref{sec:learning,sec:autonomous-learning}).


\subsection{Intelligence in animals}\label{subsec:intelligence-in-animals}

\subsection{Intelligence in humans}\label{subsec:intelligence-in-humans}


\section{Intelligence in computer science}\label{sec:intelligence-in-computer-science}

\subsection{Autonomy}\label{subsec:autonomy}

\subsection{Adapting}\label{subsec:adapting}

\subsection{Planning}\label{subsec:planning}

\subsection{Reasoning}\label{subsec:reasoning}


\section{Learning}\label{sec:learning}

\subsection{\Glsentrylong{ML}}\label{subsec:machine-learning}

\subsection{\Glsentrylong{RL}}\label{subsec:rl}

\subsection{Actively learning}\label{subsec:actively-learning}

% \subsection{\Glsentrylong{PAC} learning}\label{subsec:pac-learning}


\section{Autonomous learning}\label{sec:autonomous-learning}

\subsection{Intelligent agent}\label{subsec:intelligent-agent}

\subsection{\Glsentrylongpl{MAS}}\label{subsec:mas}

\subsection{Swarm intelligence}\label{subsec:swarm-intelligence}

% \subsection{\Glsentrylong{MARL}}\label{subsec:marl}

\subsection{General learning}\label{subsec:general-learning}
