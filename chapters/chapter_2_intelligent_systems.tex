%! Author = matteomagnini
%! Date = 05/03/25

%----------------------------------------------------------------------------------------
\chapter{Intelligent Systems}
\label{ch:intelligent-systems}
\minitoc
%----------------------------------------------------------------------------------------

\section{What is intelligence?}\label{sec:what-is-intelligence}

Intelligence is a concept that encompasses a wide range of abilities and characteristics of single \emph{individuals} or \emph{groups}.
%
From an \emph{evolutionary} perspective, intelligence characterises some animal species -- and organisms belonging to other biological kingdoms -- from simple forms of life.
%
In the history of our planet, carbon-based life evolved from unicellular organisms to multicellular organisms, and ultimately to complex organisms with specialized cells and tissues.
%
Some of these organisms (e.g., insects) developed skills -- such as \emph{navigation}, \emph{communication}, \emph{self-organisation}, \emph{adaptation}, and so on -- that \emph{are perceived} as intelligent abilities.
%
More evolved individuals (e.g., mammals) developed even more complex forms of intelligence, such as \emph{planning}, \emph{reasoning}, \emph{learning}, etc.


In much more recent history, the carbon-based life was not the only one to ``manifest'' intelligent behaviours.
%
With the invention of the computer, also machines can perform tasks that we consider intelligent.
%
To distinguish between the intelligence of living beings and that of machines, we refer to the former as \emph{natural intelligence} and to the latter as \emph{\gls{AI}}.


Intelligence originally emerged as a result of the evolutionary process and from the interaction of organisms with each others and their environment.
%
Later, we built machines that can perform tasks that require some sort of intelligent ability.
%
However, it is us -- as humans -- \emph{to attribute} the label of intelligent to someone or something and not to others.
%
Indeed, it is said that \emph{``intelligence is in the eye of the beholder''}.
%
In this sense, we can say that intelligence is not an absolute concept, but it should be considered under a relative perspective.


Giving a rigorous definition of intelligence is not a trivial task.
%
Furthermore, there is not one single shape of intelligence, but it can be declined in many different ways (e.g., \emph{emotional intelligence}, \emph{social intelligence}, \emph{spatial intelligence}, etc.).
%
In this thesis we do not stay strictly to a specific definition of intelligence, nevertheless we give a broad definition to help the reader to get more familiar some concepts that will be introduced later.
%
We adopt a definition inspired by a notorious satirical essay by Carlo Cipolla~\cite{cipolla2013allegro}.
%
In \emph{``The basic laws of human stupidity''}, the author classifies individuals into four categories based on the result of their actions:
%
\begin{itemize}
    \item \textbf{Stupid} $\rightarrow$ losses for others and for themselves;
    \item \textbf{Helpless} $\rightarrow$ benefits for others and losses for themselves;
    \item \textbf{Bandit} $\rightarrow$ losses for others and benefits for themselves;
    \item \textbf{Intelligent} $\rightarrow$ benefits for others and for themselves.
\end{itemize}
%
Now, what is a \emph{loss} or a \emph{benefit}?
%
We can see losses and benefits as the failure or success in -- fully or partially -- achieving a certain \emph{goal}.
%
With this respect we can state that:
%
\begin{definition}[Intelligent behaviour]
    \label{def:intelligence}
    an individual (or a group) has an intelligent behaviour if it is able to \textbf{perform actions} that lead to the \textbf{achievement} of a given \textbf{goal}.
\end{definition}


In this chapter, we briefly talk about intelligence in animals and humans (\Cref{subsec:intelligence-in-animals,subsec:intelligence-in-humans}).
%
Then, we introduce \gls{AI} with a focus on reasoning and learning (\Cref{sec:intelligence-in-computer-science}).
%
Finally, we focus on the core topic of this thesis: \emph{learning} and \emph{autonomous learning} (\Cref{sec:intelligence-in-computer-science,sec:autonomous-learning}).


\subsection{Intelligence in animals}\label{subsec:intelligence-in-animals}

In order to survive in their environment, animals -- but also other organisms of different kingdoms such as plants or fungi -- have developed a set of abilities and some of them are perceived as intelligent.
%
For the sake of simplicity, we will consider just to animals, but many of the concepts we will talk about can be extended to other organisms.
%
Animals have a \emph{body}, and they are \emph{situated} in the physical world.
%
Through sensory organs, they can \emph{perceive} the environment and with muscles they can \emph{interact} with it.
%
They are \emph{autonomous} individuals, i.e., they are able to perform actions without the need of an external controller.


All animals are able to keep themselves alive (\emph{self-sufficiency}) until natural death or an accident occurs.
%
In addition to self-sufficiency, animals contribute to the survival of their species generating offspring.
%
To do so, they need to navigate the environment, find food, avoid predators, reproduce, and so on.
%
\note{TODO: add more and also examples}


\subsection{Intelligence in humans}\label{subsec:intelligence-in-humans}

Humans have developed two main characteristics about intelligence that are no match for any other animals: \emph{reasoning} and \emph{learning}.

\subsubsection{Reasoning}\label{subsubsec:reasoning}
%
Logical reasoning, or simply reasoning, is a process of drawing conclusions from premises.
%
There exists three main ways of reasoning:
%
\begin{itemize}
    %
    \item \textbf{Deductive reasoning} $\rightarrow$ it is a top-down approach that starts from a general statement and derives specific conclusions.
    %
    For example, if we know that \emph{all humans are mortal} and \emph{Socrates is a human}, we can conclude that \emph{Socrates is mortal}.
    %
    Deductive reasoning can be compared to what Kahneman calls \emph{System 2} in his book \emph{Thinking, Fast and Slow}~\cite{kahneman2011thinking}.
    %
    Kahneman describes the second system as \emph{slow thinking}, which is deliberate, effortful, logical and more rational.
    %
    \item \textbf{Inductive reasoning} $\rightarrow$ it is a bottom-up approach that starts from specific observations and derives general conclusions.
    %
    For example, if we observe that \emph{the sun rises every day}, we can conclude that \emph{the sun will rise tomorrow}.
    %
    Inductive reasoning can be mapped in the \emph{System 1} -- i.e., \emph{fast thinking} -- of Kahneman, which is automatic, effortless, intuitive and emotional.
    %
    \item \textbf{Abductive reasoning} $\rightarrow$ it is a form of reasoning that starts from an observation and seeks the simplest and most likely explanation (i.e., \emph{Occam's razor}).
    %
    For example, if we observe that \emph{the grass is wet}, we can conclude that \emph{it rained last night}.
    %
    This kind of reasoning is the same that is adopted by detectives to solve crimes.
    %
    Abduction also requires the use of statistics and probabilities, and therefore it deals with uncertainty (\emph{``Once you eliminate the impossible, whatever remains, no matter how improbable, must be the truth''---Arthur Conan Doyle}).
\end{itemize}

\subsubsection{Learning}\label{subsubsec:learning}
%
\note{TODO: talk about Learning. Experience, reinforcement, share of knowledge.}



\section{Intelligence in computer science}\label{sec:intelligence-in-computer-science}

Since the automation of computation with the invention of the computer, intelligence has always been a central topic in computer science.
%
Officially, the field of \gls{AI} was born in 1956 at the Dartmouth Conference, where a group of researchers gathered to discuss the possibility of \emph{computing towards intelligence}.
%
In this section, we present how machines can perform intelligent tasks, focussing in particular on reasoning and learning.

\subsection{Reasoning}\label{subsec:reasoning}
%
\note{TODO: fill the section}

\subsection{Learning}\label{subsec:machine-learning}
%
The learning process performed by machines is called \gls{ML}.
%
\gls{ML} is a wide umbrella term that encompasses a variety of different ways of learning and different learning tasks.
%
A widely adopted definition of \gls{ML} is the one given by Tom Mitchell in 1997~\cite{DBLP:books/daglib/0087929}:
%
\begin{quote}
    \emph{``A computer program is said to learn from experience $E$ with respect to some class of tasks $T$ and performance measure $P$ if its performance at tasks in $T$, as measured by $P$, improves with experience $E$''}.
\end{quote}
%
The software component deputed to learning is referred as \emph{model}, \emph{learner} or \emph{predictor} (and possibly other names).
%
The experience $E$ can be represented as a given dataset, i.e., a collection of input data, but there could be other ways (e.g., \Cref{subsubsec:rl}).
%
The input can come in different shapes, e.g., \emph{tabular data}, \emph{images}, \emph{text}, etc.
%
Each input type is represented in different ways to be interpreted by a machine: an entry in a table is represented as a vector of features, an image is usually represented as a matrix, or a tensor, and a text can be represented as a vector of word embeddings.
%
There can be a variety of different tasks $T$ and performance measures $P$.
%
In the rest of this section, we will provide a brief overview of the most common tasks and performance measures.


The most common macro \gls{ML} tasks are:
%
\begin{itemize}
    \item \textbf{Classification} $\rightarrow$ given the input data $X={x_1, x_2, \dots, x_n}$, the task is to predict a label $y$ (a.k.a., output) from a finite set of labels $Y=\{y_1, y_2, \dots, y_m\}$.
    %
    The input is made of numerical features $x_i$ that can either be binary, categorical, or continuous.
    %
    Ultimately, the goal is to learn a prediction function $\pi^{*}: X \rightarrow Y$;
    %
    \item \textbf{Regression} $\rightarrow$ the task consists in predicting a continuous value $y$ from the input data $X$.
    %
    Similarly to classification, the goal is to learn the optimal predictor that maps the input $X$ to the output $Y$ (usually $Y=\mathbb{R}$);
    %
    \item \textbf{Clustering} $\rightarrow$ the task is to group the input data $X$ into $k$ clusters according to some strategy (e.g., usually a distance function).
    %
    A cluster is a subset of the input data $X$ that is \emph{similar to each other} and \emph{dissimilar from the data in other clusters}.
    %
    Clustering predictors can be considered as classifiers upon anonymous classes.
    %
\end{itemize}
%
To approximate $\pi^{*}$, a learning algorithm is used.
%
Depending on the type of learning algorithm, the learning process can be \emph{supervised}, \emph{unsupervised}, or \emph{reinforcement} (see \Cref{subsubsec:supervised,subsubsec:unsupervised,subsubsec:rl}).
%

\subsubsection{Supervised}\label{subsubsec:supervised}
%
A \gls{ML} process is \emph{supervised} if it is trained on a labelled dataset.
%
The training set is made of pairs $(x_i, y_i)$, where $x_i$ is the input data and $y_i$ is the corresponding label.
%
The goal of the learning process is to learn a function $f$ that maps the input data $x_i$ to the corresponding label $y_i$.
%
The function $f$ is usually a mathematical model that is trained on the training set.
%


\subsubsection{Unsupervised}\label{subsubsec:unsupervised}

\subsubsection{\Glsentrylong{RL}}\label{subsubsec:rl}

\subsection{Agents}\label{subsec:agents}
%
We will use the term software \emph{agent} (just \emph{agent} for simplicity) from now on to refer to a software process that has particular characteristics.
%
One of these is \emph{autonomy}.
%
Autonomy is the ability of an agent to act without the direct control of humans or other agents.
%
Technically speaking, an agent encapsulates its own thread of control.
%
Autonomy is not absolute, but it is a matter of degree; e.g., self-sufficiency increases the degree of autonomy of an agent.
%
Autonomy -- \emph{per se} -- does not imply intelligence.
%
For example, from a biological perspective, a cell has some degree of autonomy (e.g., it has boundaries, it interacts with the environment, etc.), but certainly we do not consider it an intelligent entity.
%
However, it is a fundamental trait of living systems and life evolved towards more and more autonomous systems.


A \emph{robot} -- i.e., an \emph{embodied agent} -- that is able to recognise when its battery is low and to recharge itself is more autonomous than a robot that needs to be recharged by a human.


\section{Autonomous learning}\label{sec:autonomous-learning}

\subsection{Intelligent agent}\label{subsec:intelligent-agent}

\subsection[Multi-agent systems]{\Glsentrylongpl{MAS}}\label{subsec:mas}

\subsection{Swarm intelligence}\label{subsec:swarm-intelligence}

% \subsection{\Glsentrylong{MARL}}\label{subsec:marl}

\subsection{General learning}\label{subsec:general-learning}
