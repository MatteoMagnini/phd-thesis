%! Author = matteomagnini
%! Date = 05/03/25

%----------------------------------------------------------------------------------------
\chapter{Introduction}
\label{ch:introduction}
\mtcaddchapter
\minitoc
%----------------------------------------------------------------------------------------

\begin{refsection}

\section{Research background and context}
\label{sec:research-background-and-context}
%
Through the course of history, humanity has experienced several socio-technological revolutions that have changed the way we live.
%
From the first industrial revolution that initiated the automation of manual labor, to the world-wide spread of computers that started the automation of processes and decision-making \sidenote{cite/talk about expert systems(?)}, we are now witnessing the \ac{AI} revolution.
%
The advent of \ac{AI} has already successfully automated cognitive tasks \sidenote{add citation to image recognition and similar}, and it is expected to go further by reaching -- and possibly surpassing -- \emph{human-level intelligence}.
%
\Ac{AI} is not a recent invention, it has been around since the 1950s.
%
The reasons why only now (in the last decade to be more precise) \ac{AI} has become ubiquitous are the presence of crucial ingredients that were missing in the past.
%
Thanks to
%
\begin{inlinelist}
    \item the enormous amount of \emph{data},
    %
    \item the improvement of \emph{memory} and \emph{computational power} -- that still follows the Moore's law --, and
    %
    \item the affordability of huge quantity of \emph{energy},
    %
\end{inlinelist}
%
\ac{AI} finally flourished again.


The first kind of \ac{AI} that was developed is \emph{symbolic}.
%
Symbolic means that there are \emph{symbols} with specific \emph{meanings} that are manipulated by algorithms.
%
Symbolic \ac{AI} follows the \emph{deductive} process of reasoning, where the system starts from a set of axioms and applies rules.
%
These kinds of \ac{AI} programs are pretty effective in well-defined domains where there are clear rules that always hold, e.g., board games,~\ac{TSP},~\acp{BWP}, etc.
%
\emph{Sub-symbolic} \ac{AI} is based on the \emph{inductive} process of reasoning.
%
Conversely to symbolic \ac{AI}, sub-symbolic \ac{AI} does not rely on symbols that have meanings for humans, but on data \emph{patterns}.
%
Programs that uses sub-symbolic \ac{AI} to solve a certain task are said to perform \ac{ML}, because a model needs to first learn from examples before being able to generalize to unseen data.
%
\sidenote{consider to add a sentence or two about other different kinds of algorithms based on different learning principles like RL}
%
Sub-symbolic models like \acp{NN} can reach \emph{super-human performance} in pre-defined tasks like image recognition, natural language processing, etc., but they require a huge amount of data and hardware resources to be trained.


The natural evolution in \ac{AI} research is to use both symbolic and sub-symbolic approaches together in order to increase the performance and face more challenging tasks.
%
This is the idea behind \emph{\ac{NeSy} \ac{AI}}, where the deductive reasoning of symbolic \ac{AI} is combined with the inductive learning of sub-symbolic models, especially \acp{NN}.
%
This branch of \ac{AI} is relatively young; the first works that combined logic rules within a \ac{NN} date back to the 90s~\cite{DBLP:conf/aaai/TowellSN90,DBLP:journals/ai/TowellS94}.
%
The last past years have been quite prolific both in the design of new \ac{NeSy} techniques and in the development of intelligent systems that use them~\cite{DBLP:journals/csur/CiattoSAMO24}.


Finally, the advent of \acp{LLM} has further transformed the landscape of \ac{AI}, offering unprecedented capabilities for natural language (and also multimodal data) generation.
%
\Acp{LLM} are huge \ac{NN} models up to \emph{hundreds of billions} of parameters that are trained on a large corpus of text data.
%
Despite the outstanding performance that \acp{LLM} have achieved in many tasks, their output is just a probability distribution over the vocabulary, therefore it is subject to errors (e.g., \emph{hallucinations}) and biases (e.g., from training data, from prompt engineering).
%
\Acp{LLM} are still a great resource for \ac{NeSy} \ac{AI} because of their performance, versatility and customizability.
%
Ultimately, the rapid progress in \ac{NeSy} \ac{AI} and the dazzling evolution of \acp{LLM} are significantly changing our world, leading to more and more intelligent systems, and possibly to the advent of the \emph{singularity}.


\section{Overview and contributions}
\label{sec:overview-and-contributions}
%
Lorem Ipsum

\section{Structure of the thesis}
\label{sec:structure-of-the-thesis}
%
Lorem Ipsum

\printbibliography[title=Reference,heading=bibintoc]

\end{refsection}