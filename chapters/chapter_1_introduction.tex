%! Author = matteomagnini
%! Date = 05/03/25

%----------------------------------------------------------------------------------------
\chapter{Introduction}
\label{ch:introduction}
\mtcaddchapter
\minitoc
%----------------------------------------------------------------------------------------

\begin{refsection}

\section{Research background and context}\label{sec:research-background-and-context}
%
Through the course of history, humanity has experienced several socio-technological revolutions that have changed the way we live.
%
From the first industrial revolution that initiated the automation of manual labor, to the world-wide spread of computers that started the automation of processes and decision-making \sidenote{cite/talk about expert systems(?)}, we are now witnessing the \ac{AI} revolution.
%
The advent of \ac{AI} has already successfully automated cognitive tasks \sidenote{add citation to image recognition and similar}, and it is expected to go further by reaching -- and possibly surpassing -- \emph{human-level intelligence}.
%
\Ac{AI} is not a recent invention, it has been around since the 1950s.
%
The reasons why only now (in the last decade to be more precise) \ac{AI} has become ubiquitous are the presence of crucial ingredients that were missing in the past.
%
Thanks to
%
\begin{inlinelist}
    \item the enormous amount of \emph{data},
    %
    \item the improvement of \emph{memory} and \emph{computational power} -- that still follows Moore's law --, and
    %
    \item the affordability of huge quantity of \emph{energy},
    %
\end{inlinelist}
%
the field of \ac{AI} finally flourished again.

\section{Overview and contributions}\label{sec:overview-and-contributions}

\section{Structure of the thesis}\label{sec:structure-of-the-thesis}

\end{refsection}