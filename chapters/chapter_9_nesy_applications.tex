%! Author = matteomagnini
%! Date = 05/03/25

%----------------------------------------------------------------------------------------
\chapter{NeSy AI for real world applications}
\label{ch:nesy-ai-for-real-world-applications}
\mtcaddchapter
\minitoc
%----------------------------------------------------------------------------------------

\section{Goals and challenges}\label{sec:nesy-ai-goals-and-challenges}

\section{Applications}\label{sec:nesy-ai-applications}
%
We now present a selection of contributions about \gls{NeSy} application in real-world scenarios.
%
These works involve the use of both \gls{SKE} and \gls{SKI} methods, as well as traditional \gls{ML} models and the recent \glspl{LLM}.



\subsection{\Gls{SKE} for explainable nutritional recommenders}\label{subsec:ske-for-explainable-nutritional-recommenders}
%
``Symbolic knowledge extraction for explainable nutritional recommenders''~\cite{DBLP:journals/cmpb/MagniniCCAO23}, published on the Journal of Computer Methods and Programs in Biomedicine, presents a novel nutritional recommender system that leverages \gls{SKE}.


\subsubsection{Nutritional recommender systems}\label{subsubsec:nutritional-recommender-systems}

Eating habits significantly impact the well-being of individuals across all age groups, highlighting the importance of developing nutritional \glspl{RS}.
%
These systems address diverse user needs, including diet programs, chronic disease management, treatment for critically ill patients, allergies, lifestyle choices (e.g., sporty, vegetarian, organic, halal), and physical activity levels~\cite{citation_needed}.
%
User preferences are represented either by expert knowledge, such as daily nutritional intake limits based on physical activity levels, or by user feedback, such as reviews on recipes~\cite{citation_needed}.
%
Recommendation items, such as food, recipes, and meals, are represented in terms of their attributes, including cuisine, category, cooking style, preparation time, and nutritional levels~\cite{citation_needed}.
%
The complexity of nutritional \glspl{RS} arises from the combination of multiple ingredients to form recipes and the diverse attributes influencing user preferences.

Classical approaches to nutritional recommendation rely on content based and collaborative filtering methods, which derive user profiles from past activities, such as ratings, clicks, reviews, and browsing history~\cite{citation_needed}.
%
Recent advancements leverage \gls{ML} techniques, including question answering over knowledge bases, recipe retrieval from visual records of ingredients, and learning recipe representations from multi-modal data~\cite{citation_needed}.
%
Despite the availability of recipes online, many are suboptimal in terms of health~\cite{citation_needed}.
%
Recent studies aim to promote healthy eating by incorporating healthiness indicators into recommendations or enhancing the visual appeal of food~\cite{citation_needed}.
%
Providing explanations for recommendations improves user trust and acceptance, as transparent systems are preferred over black-box models~\cite{citation_needed}.
%
Explainable approaches, such as explanation ontologies and multi-agent architectures, have been proposed to enhance both personalisation and explainability in nutritional \glspl{RS}~\cite{citation_needed}.

\subsubsection{Methodology}\label{subsubsec:methodology}

\begin{figure}
  \centering
  \includegraphics[width=\textwidth]{figures/ske-recommender-workflow}
  \caption{
    Data-flow perspective of a nutritional \gls{RS}.
    %
    The user interacts -- via some smart/wearable device -- with a sub-symbolic AI predictor, continuously trained to predict whether the user likes a given recipe or not.
    %
    A knowledge base is then extracted from the predictor, representing the user's preferences in a human-interpretable logic form.
    %
    Dietary prescriptions -- provided by human experts -- are transformed into the same logic form.
    %
    Conversely, databases providing information about recipies -- there including ingredients and their nutrients -- are assumed to be remotely available via the Internet.
    %
    Finally, the recommending agent exploits a logic engine, combining all such information into recommendations which are simultaneously correct (w.r.t.\ experts prescriptions), acceptable (w.r.t.\ users' preferences), and explainable.
  }
  \label{fig:ske-recommender-workflow}
\end{figure}

We propose a general architecture for nutritional \glspl{RS}, designed to provide personalised dietary recommendations aligned with specific user goals.
%
\Cref{fig:ske-recommender-workflow} illustrates the main components and data flow within the system.
%
Recommendations are generated by identifying recipes that satisfy both user preferences and expert prescriptions.

The architecture relies on three primary information sources:
%
\begin{itemize}
  \item \textbf{User preferences:} Represented sub-symbolically using a \gls{ML} predictor trained on user interaction data, such as ratings and feedback.
  %
  \item \textbf{Expert prescriptions:} Structured dietary guidelines provided by nutrition experts, detailing ingredients and nutrients required to achieve specific goals.
  %
  \item \textbf{Recipe database:} A repository containing recipes, ingredients, and their nutritional values, enabling queries based on various criteria.
\end{itemize}

User preferences are modeled as a function \( \text{appreciation}: R \to \mathbb{R} \), where \( R \) represents the set of recipes and the output indicates the user's appreciation score.
%
This score encapsulates factors such as taste, allergies, and ethical beliefs.
%
The \gls{ML} predictor adapts continually to reflect evolving user preferences~\cite{citation_needed}.

Expert prescriptions are structured representations of dietary recommendations, typically provided in tabular or quasi-natural language form.
%
Each prescription specifies the ingredients or nutrients required for a meal, along with their quantities and timing (e.g., breakfast, lunch, dinner).
%
For simplicity, prescriptions are expressed as logic formulas, enabling compatibility with the system's reasoning engine.

\paragraph{The role of logic formulas}\label{par:the-role-of-logic-formulas}

Both user preferences and expert prescriptions are represented as sets of Horn clauses.
%
This representation facilitates reasoning through logic resolution, enabling the system to identify recipes that satisfy both user preferences and expert prescriptions.
%
For example, given the query \( \text{likes}(R) \land \text{should\_eat}(R) \), the system can compute admissible recipes or determine if no suitable options exist.

\paragraph{The role of \gls{SKE}}\label{par:the-role-of-ske}

To bridge the gap between sub-symbolic user preferences and symbolic reasoning, the architecture employs \gls{SKE}.
%
This step extracts symbolic knowledge in the form of Horn clauses from the trained \gls{ML} predictor.
%
Implementers are free to choose the most suitable \gls{SKE} algorithm, provided it outputs clauses compatible with the reasoning engine.
%
This integration ensures that recommendations are both explainable and aligned with user preferences.


\subsubsection{Validation}\label{subsubsec:validation-ske-nutrition}
%
To validate the proposed architecture, we conducted a series of experiments to assess its effectiveness in nutritional \glspl{RS}.
%
The source code and instructions for reproducing these experiments are available online~\cite{citation_needed}.
%
The experiments involved four main steps:
%
\begin{enumerate}
  \item Generating synthetic datasets to simulate a single user's food preferences.
  %
  \item Training a \gls{ML} predictor, specifically a neural network, to predict whether a recipe would be liked by the user.
  %
  \item Applying a \gls{SKE} algorithm to extract symbolic knowledge that represents the decision-making behavior of the predictor.
  %
  \item Evaluating the system's ability to recommend recipes that align with both user preferences and expert nutritional prescriptions.
\end{enumerate}
%
Details regarding data selection, synthesis, and experimental procedures are provided in the following subsections.

\paragraph{Datasets}\label{par:datasets-ske-nutrition}
%
We utilized a public dataset of recipes~\cite{citation_needed} and generated 12 synthetic datasets to represent the preferences of imaginary users.
%
The recipe dataset consists of four files:
%
\begin{itemize}
  \item \textbf{Recipe Details:} Contains recipe IDs, titles, sources, and cuisines.
  %
  \item \textbf{Ingredients:} Lists basic ingredients with aliases, synonyms, entity IDs, and categories.
  %
  \item \textbf{Compound Ingredients:} Includes compound ingredients with their constituent components and categories.
  %
  \item \textbf{Recipe-Ingredients Aliases:} Maps recipes to their ingredients using aliases and entity IDs.
\end{itemize}
%
The dataset includes 929 unique basic ingredients and 103 compound ingredients, categorized into 21 groups (e.g., additive, bakery, beverage).
%
Recipes with at least one ingredient total 45,749, while recipes without ingredients are excluded.

Synthetic datasets were created in two steps.
%
First, unconditional preferences for individual ingredients were generated based on predefined user profiles.
%
Each profile specifies ranges of values for ingredients or categories (e.g., vegetables, meat), which are used to compute likelihoods via uniform distribution.
%
Second, recipe likability labels (like/dislike) were generated based on the likelihood values of the recipe's ingredients.
%
The synthesis process ensures no real personal data is used, avoiding ethical or privacy concerns.

\paragraph{Learning User Preferences via Sub-Symbolic Predictors}\label{par:learning-user-preferences}
%
User preferences were modeled using fully-connected neural networks.
%
Each network consists of one input layer, two hidden layers, and one output layer, with 1032, 16, 8, and 1 neurons, respectively.
%
The activation functions for the input and hidden layers are ReLU, while the output layer uses a sigmoid function.
%
The input to the network is a tensor representing the presence of 1032 ingredients in a recipe, and the output is a scalar indicating the likelihood of user appreciation.

A separate neural network was trained for each of the 12 synthetic users.
%
Training was performed on half of the dataset (22,874 records), while the remaining half was used for testing.
%
The training process lasted 20 epochs with a batch size of 32.
%
The average accuracy achieved on the test set was 85.6\%, with precision computed as the ratio of true positives to total positive predictions~\cite{citation_needed}.
%
Precision is critical for systems where identifying true positives (liked recipes) is more important than minimizing false positives.

\paragraph{Extracting User Preferences via \Gls{SKE}}\label{par:extracting-user-preferences}
%
\begin{table}%[!h]
    \centering
    \caption[Datasets statistics for each users on test sets.]{
        Datasets statistics for each users on test sets.
        %
        The second column shows the \emph{like} class ratio.
        %
        Third and fourth columns describe the accuracy and precision score of trained neural networks.
        %
        Fifth and sixth columns describe the accuracy and precision score of the extracted rules.
        %
        The last column shows the fidelity of the extracted rules w.r.t.\ the NN.
    }
    % \begin{adjustbox}{width=\linewidth,center}
        \begin{tabular}{l||r|r|r|r|r|r}
            \textbf{users} & \textbf{liked ratio} & \textbf{net acc.} & \textbf{net prec.} & \textbf{r. acc.} & \textbf{r. prec.} & \textbf{r. fid.}\\
            \hline\hline
            user 1 & 0.336 & 0.9233 & 0.8855 & 0.864 & 0.8228 & 0.873\\
            user 2 & 0.2842 & 0.9714 & 0.9564 & 0.794 & 0.7688 & 0.7972\\
            user 3 & 0.3941 & 0.8503 & 0.8221 & 0.7726 & 0.7136 & 0.7955\\
            user 4 & 0.4502 & 0.8364 & 0.8165 & 0.7562 & 0.7145 & 0.7813\\
            user 5 & 0.3022 & 0.9621 & 0.9478 & 0.8381 & 0.8301 & 0.842\\
            user 6 & 0.2719 & 0.9722 & 0.9514 & 0.7997 & 0.7109 & 0.8034\\
            user 7 & 0.3504 & 0.8916 & 0.8369 & 0.7782 & 0.7311 & 0.7901\\
            user 8 & 0.449 & 0.7782 & 0.7594 & 0.6868 & 0.674 & 0.7283\\
            user 9 & 0.393 & 0.8085 & 0.7783 & 0.7493 & 0.7524 & 0.803\\
            user 10 & 0.4734 & 0.7381 & 0.7582 & 0.6862 & 0.7452 & 0.7616\\
            user 11 & 0.4359 & 0.7708 & 0.7863 & 0.7098 & 0.7426 & 0.7804\\
            user 12 & 0.4353 & 0.7772 & 0.7979 & 0.7047 & 0.7467 & 0.7814\\
            \hline\hline
            average & 0.3813 & 0.8567 & 0.8414 & 0.7616 & 0.7461 & 0.7948\\
        \end{tabular}
    % \end{adjustbox}
\label{tab:net-rules-stats}
\end{table}
%
Symbolic knowledge was extracted from the trained \gls{ML} predictors using the CART algorithm~\cite{citation_needed}.
%
This algorithm generates decision trees, which are converted into logic rules.
%
Each path from the root to a leaf in the decision tree corresponds to a rule, where nodes represent logical conditions (e.g., presence or absence of ingredients) and leaves denote the predicted class.

The maximum number of leaves was set to \( R = 50 \), and the maximum depth of the decision tree was limited to \( D = 10 \).
%
These constraints balance computational efficiency and interpretability, ensuring the rules are concise and comprehensible.
%
For example, extracted rules for a user might include:
%
\begin{align*}
  \text{likes}(R) &\leftarrow \text{has\_no}(R, \text{egg}) \land \text{has\_no}(R, \text{pepper}) \land \text{has}(R, \text{almond}), \\
  \neg \text{likes}(R) &\leftarrow \text{has}(R, \text{egg}) \land \text{has}(R, \text{parsley}).
\end{align*}
%
These rules approximate the decision-making process of the neural network, enabling explainability.

We constrain the output rules for two main reasons.
%
First, limiting the growth of the decision tree (\gls{DT}) reduces computational complexity.
%
Given \(N\) binary features, such as ingredients, the maximum depth of the \gls{DT} is \(N + 1\), and the maximum number of leaves is \(2^N\).
%
For \(N = 1032\), this would be computationally infeasible.
%
Second, we aim to ensure the extracted rules remain interpretable.
%
Rules with excessively long right-hand sides or an overwhelming number of conditions are difficult for humans to read and understand.
%
This trade-off prioritizes interpretability over performance metrics, such as accuracy and precision, which is essential for explaining why a prescription may or may not be suitable for a user.

\Cref{tab:net-rules-stats} summarizes the accuracy of the extracted rules on the test sets.
%
It also reports the fidelity of the rules with respect to the sub-symbolic predictors.
%
Fidelity is computed similarly to accuracy but compares the predictions of the extracted rules against the class values predicted by the \gls{ML} predictor, rather than the ground truth~\cite{citation_needed}.
%
In other words, fidelity measures how closely the extracted rules replicate the behavior of the neural networks.


\paragraph{Proposed Recipes}\label{par:proposed-recipes}
%
\begin{table}%[!h]
    \centering
    \caption[Precision of proposed recipes per user and prescription]{%
        Precision values of the algorithm per user and prescription. Precision values denote actually liked recipes (i.e., true positive) over the all proposed recipes (i.e., true positive plus false positive) obtained by prescriptions and extracted rules.
    }
    % \begin{adjustbox}{width=\linewidth,center}
        \begin{tabular}{l||r|r|r|r|r|r||r}
            \textbf{users} & \textbf{p. 1} & \textbf{p. 2} & \textbf{p. 3} & \textbf{p. 4} & \textbf{p. 5} & \textbf{p. 6} & \textbf{average}\\
            \hline\hline
            user 1 & 0.831 & 0.8 & 0.8621 & 0.6667 & 0.6875 & 0.7857 & 0.7722\\
            user 2 & 0.6338 & 0.2979 & 0.8 & 0.7083 & 0.9268 & 0.7727 & 0.6899\\
            user 3 & 0.7625 & 0.4333 & 0.7297 & 0.75 & 0.6111 & 0.6604 & 0.6578\\
            user 4 & 0.75 & 0.6 & 0.8387 & 0.65 & 0.5435 & 0.7755 & 0.693\\
            user 5 & 0.8571 & 0.7667 & 0.95 & 0.7083 & 0.8182 & 0.8095 & 0.8183\\
            user 6 & 0.6 & 0.5116 & 0.8636 & 0.8261 & 0.55 & 0.619 & 0.6617\\
            user 7 & 0.7625 & 0.4667 & 0.1852 & 0.8276 & 0.7391 & 0.7826 & 0.6273\\
            user 8 & 0.85 & 0.75 & 0.9048 & 0.5758 & 0.6667 & 0.7872 & 0.7558\\
            user 9 & 0.6316 & 0.9 & 0.8929 & 0.8085 & 0.8936 & 0.6809 & 0.8012\\
            user 10 & 0.875 & 0.66 & 0.7692 & 0.7879 & 0.717 & 0.7692 & 0.763\\
            user 11 & 0.8625 & 0.84 & 0.963 & 0.7391 & 0.64 & 0.8113 & 0.8093\\
            user 12 & 0.7875 & 0.8333 & 0.9024 & 0.8108 & 0.7037 & 0.8776 & 0.8192\\
            % \hline
            % \hline
            % average & 0.767 & 0.655 & 0.8051 & 0.7383 & 0.7081 & 0.761 & 0.7391\\
        \end{tabular}
        % \end{adjustbox}
    \label{tab:proposed-recipes-stats}
\end{table}


%
\begin{table}%[!h]
    \centering
    \caption[Precision of proposed recipes with NN]{%
        Precision values of the algorithm per user and prescription. Note that precision values denote actually liked recipes (i.e., true positive) over the all proposed recipes (i.e., true positive plus false positive) obtained by prescriptions and NN.
    }
    % \begin{adjustbox}{width=\linewidth,center}
        \begin{tabular}{l||r|r|r|r|r|r||r}
           \textbf{users} & \textbf{p. 1} & \textbf{p. 2} & \textbf{p. 3} & \textbf{p. 4} & \textbf{p. 5} & \textbf{p. 6} & \textbf{average}\\
            \hline\hline
            user 1 & 0.8267 & 0.8333 & 1.0 & 0.9474 & 0.8696 & 0.8913 & 0.8947\\
            user 2 & 0.95 & 0.9474 & 0.931 & 1.0 & 0.95 & 0.94 & 0.9531\\
            user 3 & 0.7875 & 0.7797 & 0.7273 & 0.8478 & 0.88 & 0.8163 & 0.8064\\
            user 4 & 0.775 & 0.7833 & 0.871 & 0.8158 & 0.7442 & 0.7447 & 0.789\\
            user 5 & 0.9265 & 0.9333 & 0.9565 & 0.9474 & 0.9524 & 0.9375 & 0.9423\\
            user 6 & 0.9375 & 0.9833 & 1.0 & 0.9615 & 0.9592 & 0.9583 & 0.9666\\
            user 7 & 0.8 & 0.8667 & 0.9 & 0.8571 & 0.8723 & 0.8913 & 0.8646\\
            user 8 & 0.8625 & 0.7833 & 0.8837 & 0.7 & 0.8077 & 0.8065 & 0.8073\\
            user 9 & 0.7875 & 0.7833 & 0.9118 & 0.8095 & 0.6786 & 0.6842 & 0.7758\\
            user 10 & 0.8125 & 0.5667 & 0.8049 & 0.7632 & 0.7959 & 0.84 & 0.7639\\
            user 11 & 0.8875 & 0.7667 & 0.7674 & 0.8723 & 0.7917 & 0.7931 & 0.8131\\
            user 12 & 0.825 & 0.8667 & 0.8049 & 0.68 & 0.8246 & 0.8364 & 0.8063\\
            % \hline
            % \hline
            % average & 0.8482 & 0.8245 & 0.8799 & 0.8502 & 0.8439 & 0.845 & 0.8486\\
        \end{tabular}
        % \end{adjustbox}
    \label{tab:proposed-recipes-stats-nn}
\end{table}


%
The experiments aim to evaluate how user preferences and expert prescriptions are combined to recommend recipes.
%
We rely on sets of logic rules to express domain-expert prescriptions, ensuring consistency with the formalism used for user preferences.
%
In total, six prescriptions are defined, corresponding to three days with two meals per day.
%
For each meal, multiple rules (ranging from two to four) are specified, one for each dish.

\begin{prescriptions}
  \item First day, lunch: ``Rice with vegetables.'' Ingredients include 80 grams of raw rice, 35 grams of raw lentils, 120 grams of raw chicken, 120 grams of mixed vegetables, garlic, herbs, 2 teaspoons of olive oil, and 1 orange.
  %
  \begin{align*}
    \begin{array}{rcl}
    \text{should\_eat}(R) & \leftarrow & \text{has}(R, \text{chicken}) \land \text{has}(R, \text{rice}), \\
    \text{should\_eat}(R) & \leftarrow & \text{has}(R, \text{lentils}), \\
    \text{should\_eat}(R) & \leftarrow & \text{has}(R, \text{orange}), \\
    \text{should\_eat}(R) & \leftarrow & \text{has}(R, \text{garlic})\\
                          & \land & \text{has}(R, X) \land \text{has}(R, Y) \land \text{has}(R, Z)\\
                          & \land & \text{vegetable}(X) \land \text{herb}(Y) \land \text{essential\_oil}(Z).
    \end{array}
  \end{align*}
  %
  \item First day, dinner: ``Burger and grilled vegetables.'' Ingredients include 90 grams of beef, 80 grams of bread, 120 grams of vegetables, 1 teaspoon of oil, and 1 cup of strawberries.
  %
  \begin{align*}
    \begin{array}{rcl}
      \text{should\_eat}(R) &\leftarrow& \text{has}(R, \text{beef}) \land \text{has}(R, \text{bread}), \\
      \text{should\_eat}(R) &\leftarrow& \text{has}(R, \text{strawberry}), \\
      \text{should\_eat}(R) &\leftarrow& \text{has}(R, X) \land \text{has}(R, Y) \\
                            &\land& \text{vegetable}(X) \land \text{essential\_oil}(Y).
    \end{array}
  \end{align*}
  %
  \item Second day, lunch: ``Tuna salad.'' Ingredients include 120 grams of tuna, 120 grams of vegetables, 1 teaspoon of olive oil, 80 grams of bread, 35 grams of raw beans, and 1 cup of blueberries.
  %
  \begin{align*}
    \begin{array}{rcl}
      \text{should\_eat}(R) &\leftarrow& \text{has}(R, \text{bread}) \land \text{has}(R, \text{beans}), \\
      \text{should\_eat}(R) &\leftarrow& \text{has}(R, \text{tuna}) \\
                            &\land& \text{has}(R, X) \land \text{has}(R, Y) \\
                            &\land& \text{vegetable}(X) \land \text{herb}(X) \land \text{essential\_oil}(Y), \\
      \text{should\_eat}(R) &\leftarrow& \text{has}(R, \text{blueberry}).
    \end{array}
  \end{align*}
  %
  \item Second day, dinner: ``Chicken with mustard and lemon juice.'' Ingredients include 90 grams of chicken, 120 grams of vegetables, 80 grams of raw pasta, 1 teaspoon of olive oil, mustard, lemon juice, and 1 cup of clementines.
  %
  \begin{align*}
    \begin{array}{rcl}
      \text{should\_eat}(R) &\leftarrow& \text{has}(R, \text{chicken}) \land \text{has}(R, \text{mustard}) \land \text{has}(R, \text{lemon\_juice}), \\
      \text{should\_eat}(R) &\leftarrow& \text{has}(R, \text{pasta}), \\
                            &\land& \text{has}(R, X) \land \text{essential\_oil}(X), \\
      \text{should\_eat}(R) &\leftarrow& \text{has}(R, X) \land \text{vegetable}(X), \\
      \text{should\_eat}(R) &\leftarrow& \text{has}(R, \text{citrus\_fruits}).
    \end{array}
  \end{align*}
  %
  \item Third day, lunch: ``Salmon with potatoes.'' Ingredients include 120 grams of salmon, 240 grams of cooked potatoes, 120 grams of vegetables, 1 teaspoon of butter, and 1 pear.
  %
  \begin{align*}
    \begin{array}{rcl}
      \text{should\_eat}(R) &\leftarrow& \text{has}(R, \text{compound\_salmon}) \land \text{has}(R, \text{potato}), \\
      \text{should\_eat}(R) &\leftarrow& \text{has}(R, \text{pear}), \\
      \text{should\_eat}(R) &\leftarrow& \text{has}(R, \text{butter}) \\
                            &\land& \text{has}(R, X) \land \text{vegetable}(X).
    \end{array}
  \end{align*}
  %
  \item Third day, dinner: ``Turkey in papillote.'' Ingredients include 90 grams of turkey, 1 teaspoon of olive oil, 120 grams of vegetables, 35 grams of raw gram beans, 80 grams of raw wholegrain rice, and 1 orange.
  %
  \begin{align*}
    \begin{array}{rcl}
      \text{should\_eat}(R) & \leftarrow & \text{has}(R, \text{gram\_bean}), \\
      \text{should\_eat}(R) & \leftarrow & \text{has}(R, \text{rice}), \\
      \text{should\_eat}(R) & \leftarrow & \text{has}(R, \text{orange}), \\
      \text{should\_eat}(R) & \leftarrow & \text{has}(R, \text{turkey}) \\
                            & \land & \text{has}(R, X) \land \text{has}(R, Y) \\
                            & \land & \text{vegetable}(X) \land \text{essential\_oil}(Y).
    \end{array}
  \end{align*}
\end{prescriptions}
%
For each user and prescription, recipes are computed based on the logic rules.
%
Precision is calculated as the ratio of recipes liked by users to the total number of proposed recipes.
%
Corner cases, where no recipes are recommended due to conflicting preferences and prescriptions, are resolved by adjusting prescriptions to better align with user preferences.


\subsubsection{Results and Discussion}\label{subsubsec:results-and-discussion}
%
To ensure realistic experiments, we adopted criteria to generate synthetic datasets representing user preferences.
%
We avoided synthesizing users with trivial rules, such as always liking a specific ingredient, which would result in predictable recommendations.
%
To address this, we introduced noise into the dataset synthesis process, assigning preference values within distributions and stochastically labeling classes.
%
This approach discourages oversimplified logic rules and mimics real-life scenarios where diverse ingredient combinations influence user preferences in complex ways.
%
\Cref{tab:net-rules-stats} reports the accuracy of the neural networks trained to predict user preferences, alongside the accuracy and fidelity of the extracted logic rules.
%
The accuracy of individual networks ranges from \(0.74\) to \(0.97\), with a mean value of approximately \(0.86\).
%
This variability reflects differences in user profiles, as some preferences are inherently more predictable than others.
%
The accuracy of the extracted rules ranges from \(0.68\) to \(0.86\), with a mean value of \(0.76\).
%
The mean difference of \(0.095\) between network accuracy and rule accuracy is expected, given the constraints imposed on decision tree depth during the extraction process.
%
Similar observations apply to precision measures.
%
It is important to note that extracted rules approximate the decision-making process of the neural networks and cannot outperform the original models.

\Cref{tab:proposed-recipes-stats-nn} summarizes the precision obtained during the recommendation phase using prescriptions and extracted rules.
%
The mean precision value across all experiments is approximately \(0.74\), which is close to the average precision of the rules reported in \Cref{tab:net-rules-stats}.
%
Applying the Student's \(t\)-test to the precision values in \Cref{tab:net-rules-stats} (``r. prec.'' column) and \Cref{tab:proposed-recipes-stats} (``average'' column) yields a \(p\)-value of \(0.386\).
%
This indicates no statistical difference between the two distributions.
%
In other words, recommending recipes liked by users from the entire dataset is as effective as recommending recipes prescribed by human experts.

\Cref{tab:proposed-recipes-stats-nn} compares the precision obtained using prescriptions and sub-symbolic predictors.
%
Similar statistical analysis yields a \(p\)-value of \(0.403\), leading to the same conclusion: the recommendation process is equally effective for prescribed liked recipes and liked recipes predicted by neural networks.

In summary, experiments demonstrate that sub-symbolic predictors outperform symbolic predictors in terms of overall precision for personalized food recommendations.
%
However, the primary goal of the framework is not to achieve higher performance compared to sub-symbolic predictors or existing systems.
%
Instead, the focus is on enhancing explainability, enabling users and experts to understand why certain recipes are recommended or rejected.
%
Extracted rules provide insights that allow experts to adjust prescriptions to better align with user preferences.
%
Despite lower performance compared to neural networks, the rules remain acceptable in real-world scenarios.
%
For instance, consider the recipe ``Shakkara (Sweet) Pongal'' (recipe ID \(4,055\)), which is liked by User~1.
%
This recipe contains ingredients such as basmati rice, butter, camphor, cardamom, cashew nuts, lentils, milk, raisins, and sugar.
%
The recommendation is justified by the presence of milk and sugar, as indicated by the extracted rules for User~1.
%
Conversely, the recipe ``Lasagna Spinach Roll-Ups'' (recipe ID \(10,815\)) is disliked due to the presence of eggs and pepper.
%
The symbolic approach adds value by providing explainability, allowing motivations for recommendations to be derived systematically.



\subsection{A general-purpose protocol for multi-agent based explanations}\label{subsec:a-general-purpose-protocol-for-multi-agent-based-explanations}

\subsection{\Glsentrylong{NeSy} \Gls{AI} for supporting chronic disease diagnosis and monitoring}\label{subsec:nesy-ai-for-supporting-chronic-disease-diagnosis-and-monitoring}

%\subsection{\Gls{LLM}-based solutions for healthcare chatbots: a comparative analysis}\label{subsec:llm-based-solutions-for-healthcare-chatbots-a-comparative-analysis}

%\subsection{Open-source small language models for personal medical assistant chatbots}\label{subsec:open-source-small-language-models-for-personal-medical-assistant-chatbots}

\subsection{Applying \Gls{RAG} on open \Glspl{LLM} for a medical chatbot supporting hypertensive patients}\label{subsec:applying-rag-on-open-llm-for-a-medical-chatbot-supporting-hypertensive-patients}