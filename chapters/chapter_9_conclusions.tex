%! Author = matteomagnini
%! Date = 05/03/25

%----------------------------------------------------------------------------------------
\chapter{Conclusions}
\label{ch:conclusions}
\minitoc
%----------------------------------------------------------------------------------------

This thesis situates itself at the intersection of symbolic and sub-symbolic \gls{AI}, focusing on the integration of both to enhance the capabilities of intelligent systems.
%
The foundation of this work lies in classical \gls{AI} techniques and \gls{KR} such as computational logic and ontologies (\Cref{ch:intelligent-systems}).
%
An \gls{SLR} was conducted to explore the current landscape of \gls{SKI} and \gls{SKE} techniques (\Cref{ch:nesy-ai}) defining a taxonomy to categorize existing approaches and identifying gaps in the literature.
%
A considerable amount of the work carried out in this thesis contributed to the design and development of \gls{SKI} methods, a novel \gls{SKI} framework (\gls{PSyKI}), and evaluation metrics (\Cref{ch:ski-methods-and-contributions,ch:psyki}).
%
Also, studies about \gls{AI} and society, in particular related to the fairness of \gls{AI} systems through \gls{SKI} methods, were conducted (\Cref{ch:fairness-through-ski}).
%
Real-world applications of \gls{SKI} techniques were explored, demonstrating their effectiveness in various domains (\Cref{ch:nesy-ai-for-real-world-applications}).
%
Finally, because the ultimate goal of this work is to design and implement intelligent systems that can autonomously learn knowledge about a given domain, solutions based on \gls{SKI} and \gls{SKE} techniques were proposed and evaluated (\Cref{ch:autonomous-learning-systems}).


\section{Discussion}\label{sec:discussion}

\subsection*{About the relevance of \gls{SKI} and \gls{SKE} techniques}
%
Regarding the research question \Cref{itm:rq0}, introduced in \Cref{ch:introduction}, we conducted an \gls{SLR} to investigate the current state of \gls{SKI} and \gls{SKE} techniques (\Cref{ch:nesy-ai}).
%
The \gls{SLR} investigated \emph{249} relevant papers, identifying \emph{117} distinct \gls{SKI} works and \emph{132} \gls{SKE} methods.
%
These high numbers, along with the increasing trend of publications in recent years, indicate that \gls{SKI} and \gls{SKE} are active research areas within the \gls{AI} community.
%
In the course of other chapters -- e.g., when presenting the possibility to use \gls{SKI} techniques to enhance the fairness of \gls{AI} systems in \Cref{ch:fairness-through-ski} and when showing real-world applications in \Cref{ch:nesy-ai-for-real-world-applications} -- we further demonstrated the relevance of \gls{SKI} and \gls{SKE} techniques in addressing contemporary challenges in \gls{AI}.


\subsection*{About the characteristics of \gls{SKI} and \gls{SKE} techniques}
%
The \gls{SLR} also provided insights into the characteristics of existing \gls{SKI} and \gls{SKE} techniques posed by the research question \Cref{itm:rq1}.
%
A comprehensive taxonomy was developed to categorize the techniques based on various dimensions.


For \gls{SKI}, the dimensions that characterise most the methods are \emph{input knowledge type}, \emph{sub-symbolic model target}, and \emph{injection strategy}.
%
We observed that \gls{SKI} techniques predominantly utilize structured knowledge representations, such as ontologies and knowledge graphs, to inject knowledge into sub-symbolic models.
%
Also logic rules are widely used.
%
Regarding the target sub-symbolic models, \gls{SKI} methods mainly focus on \glspl{NN} of any kind, with a predominance of feed-forward architectures.
%
Finally, the injection strategies are quite balanced among the three main categories: \emph{structuring}, \emph{embedding}, and \emph{guided learning}.


For \gls{SKE}, the most characterising dimensions are \emph{output knowledge shape}, \emph{output knowledge expressiveness}, and \emph{translucency}.
%
\gls{SKE} techniques predominantly generate knowledge in the form of logic rules, with \gls{DT} being the second most common shape.
%
The expressiveness of the extracted knowledge is often limited to propositional logic, with fewer methods producing first-order logic or more complex representations.
%
Regarding translucency, \gls{SKE} methods are fairly evenly distributed among \emph{pedagogical} and \emph{decompositional} approaches.


\subsection*{Measuring the effectiveness of \gls{SKI} and \gls{SKE} techniques}


\subsection*{On the use of \gls{SKI} and \gls{SKE} techniques to design \gls{NeSy} \gls{AI} systems}


\section{Future work}\label{sec:future-work}