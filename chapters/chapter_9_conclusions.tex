%! Author = matteomagnini
%! Date = 05/03/25

%----------------------------------------------------------------------------------------
\chapter{Conclusions}
\label{ch:conclusions}
\minitoc
%----------------------------------------------------------------------------------------

This thesis situates itself at the intersection of symbolic and sub-symbolic \gls{AI}, focusing on the integration of both to enhance the capabilities of intelligent systems.
%
The foundation of this work lies in classical \gls{AI} techniques and \gls{KR} such as computational logic and ontologies (\Cref{ch:intelligent-systems}).
%
An \gls{SLR} was conducted to explore the current landscape of \gls{SKI} and \gls{SKE} techniques (\Cref{ch:nesy-ai}) defining a taxonomy to categorize existing approaches and identifying gaps in the literature.
%
A considerable amount of the work carried out in this thesis contributed to the design and development of \gls{SKI} methods, a novel \gls{SKI} framework (\gls{PSyKI}), and evaluation metrics (\Cref{ch:ski-methods-and-contributions,ch:psyki}).
%
Also, studies about \gls{AI} and society, in particular related to the fairness of \gls{AI} systems through \gls{SKI} methods, were conducted (\Cref{ch:fairness-through-ski}).
%
Real-world applications of \gls{SKI} techniques were explored, demonstrating their effectiveness in various domains (\Cref{ch:nesy-ai-for-real-world-applications}).
%
Finally, because the ultimate goal of this work is to design and implement intelligent systems that can autonomously learn knowledge about a given domain, solutions based on \gls{SKI} and \gls{SKE} techniques were proposed and evaluated (\Cref{ch:autonomous-learning-systems}).


\section{Discussion}\label{sec:discussion}

\section{Future work}\label{sec:future-work}