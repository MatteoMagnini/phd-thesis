% !TeX spellcheck = en_GB
% !TeX root = ../phd-thesis.tex

\begin{table}
    \centering
    \begin{tabular}{l|r||cl|r}
        \textbf{Formula} & \textbf{C. interpretation} & & \textbf{Formula} & \textbf{C. interpretation}
        \\
        \hline\hline
        $\llbracket\neg \phi\rrbracket$ & $\eta(1 - \llbracket\phi\rrbracket)$ & & $\llbracket\phi \le \psi\rrbracket$  & $\eta(1 + \llbracket \psi \rrbracket - \llbracket \phi \rrbracket)$  % Negation % Less equal
        \\
        $\llbracket\phi  \wedge \psi\rrbracket$ &  $\eta(min(\llbracket\phi\rrbracket, \llbracket\psi\rrbracket))$ & &  $\llbracket \pred{class}(\bar{X}, \const{y}_i) \leftarrow \psi \rrbracket$ & $\llbracket \psi \rrbracket^{*}$ % Conjunction % Class
        \\
        $\llbracket\phi  \vee \psi\rrbracket$ & $\eta(max(\llbracket\phi\rrbracket, \llbracket\psi\rrbracket))$ & & $\llbracket \text{expr}(\bar{X}) \rrbracket$ & $\text{expr}(\llbracket\bar{X}\rrbracket)$ % Disjunction
        \\
        $\llbracket\phi = \psi\rrbracket$ & $\eta(\llbracket\neg( \phi \ne \psi )\rrbracket )$ & &$\llbracket \mathtt{true} \rrbracket$ & $1$ % Equal
        \\
        $\llbracket\phi \ne \psi\rrbracket$ & $\eta(|\llbracket\phi\rrbracket-\llbracket\psi\rrbracket|)$ & & $\llbracket \mathtt{false} \rrbracket$ & $0$ % Not Equal
        \\
        $\llbracket\phi > \psi\rrbracket$ & $\eta(max(0, \frac{1}{2} + \llbracket\phi\rrbracket - \llbracket\psi\rrbracket))$ & & $\llbracket X \rrbracket$ & $x$ % Greater
        \\
        $\llbracket\phi \ge \psi\rrbracket$  & $\eta(1 + \llbracket \phi \rrbracket - \llbracket \psi \rrbracket)$ & & $\llbracket \const{k} \rrbracket$ & $k$ % Greater Equal
        \\
        $\llbracket\phi < \psi\rrbracket$  &  $\eta(max(0, \frac{1}{2} + \llbracket\psi\rrbracket - \llbracket\phi\rrbracket))$ & & $\llbracket \pred{p}(\bar{X}) \rrbracket^{**}$ & $\llbracket \psi_1 \vee \ldots \vee \psi_k \rrbracket$ % Less 
    \end{tabular}
    \begin{center}\scriptsize
        $^{*}$ encodes the value for the $i^{th}$ output
        \\
        \smallskip
        $^{**}$ assuming $p$ is defined by $k$ clauses of the form:
        \\
        $\pred{p}(\bar{X}) \leftarrow \psi_1,\ \ldots,\ \pred{p}(\bar{X}) \leftarrow \psi_k$
    \end{center}
    \caption{
        Logic formulas' encoding into real-valued functions.
        %
        There, $X$ is a logic variable, while $x$ is the corresponding real-valued variable, whereas is $\bar{X}$ a tuple of logic variables.
        %
        Similarly, $\const{k}$ is a numeric constant, and $k$ is the corresponding real value, whereas $\const{k}_i$ is the constant denoting the $i^{th}$ class of a classification problem.
        %
        Finally, $\text{expr}(\bar{X})$ is an arithmetic expression involving the variables in $\bar{X}$.
        %
        The $\eta$ function is a scaling function described in \Cref{eq:eta-function}.
    }
    \label{tab:kins-logic-formulae}
\end{table}

