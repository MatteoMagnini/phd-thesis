% !TeX root = ../phd-thesis.tex
\begin{table}
    \centering
     \resizebox{\columnwidth}{!}{
         \begin{tabular}{l||rrr|rrr|rrr}
             \toprule
             \multirow{2}{*}{Dataset} &  \multicolumn{3}{c|}{$R_{N, D}(\mathcal{I})$ drop} &  \multicolumn{3}{c|}{$R_{N, D}(\mathcal{I})$ noise} &  \multicolumn{3}{c}{$R_{N, D}(\mathcal{I})$ flip}\\
             \cmidrule{2-10}
             & KINS &  KILL &  KBANN & KINS &  KILL &  KBANN & KINS &  KILL &  KBANN\\
             \midrule
             BCW    & \textbf{1.0493} & \textbf{1.0318} & \textbf{1.0382}  & 0.9960 & 0.9985 & \textbf{1.0109} & 0.9994 & \textbf{1.0184} & 0.9520\\
             PSJGS & \textbf{1.0045} & 0.9968 & 0.8425 & 0.9950 & 0.9984 & \textbf{1.0145} & 0.9962 &  \textbf{1.0026} & \textbf{1.6749} \\
             CI   &  0.9998 & \textbf{1.0039} & \textbf{1.0043} & 0.9992 & \textbf{1.0012} &  0.9965 & 0.9897 & \textbf{1.1703} & 0.9815 \\
             \bottomrule
         \end{tabular}
     }
    \caption[Robustness relative scores for drop, noise and label flip perturbations]{
        %
        Robustness relative scores $R_{N, D}(\mathcal{I})$ for the three perturbation strategies: drop, noise and label flip.
        %
        Bold numbers are the ones greater than 1 (i.e., the educated model is more robust than the uneducated one).
    }
    \label{tab:robustness}
\end{table}

